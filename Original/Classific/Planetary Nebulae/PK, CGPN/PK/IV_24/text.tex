%\documentclass[12pt,a4paper,greek,english,german,headinclude]{scrartcl}
\documentclass[12pt,a4paper,greek,english,german,onecolumn,headinclude]{article}
%\documentclass[12pt,a4paper,greek,english,german,headinclude]{scrbook}
\usepackage{array}
\usepackage{tabularx}
\usepackage{longtable}
\usepackage{hhline}
\usepackage{afterpage}
\usepackage{dcolumn}
\usepackage{amsfonts}
\usepackage{amsmath}
\usepackage{amssymb}
\usepackage{graphics}
\usepackage{graphicx}
\usepackage{epsfig}
\usepackage{flafter}
\usepackage{babel}
\usepackage[T1]{fontenc}
\usepackage[cp850]{inputenc}
\usepackage{textcomp}
\usepackage{wrapfig}
\unitlength 1.0cm
\topmargin -2.cm
\oddsidemargin -0.3cm
\textwidth  16.5cm
\textheight 24.0cm
\newcommand{\url}[1]{\texttt{#1}} 
%Web-Adresse in Schreibmaschinenschrift, Aufruf durch \url{http:...}
%\newcommand{astroversion}[1]{%
% \ifthenelse{\equal{#1}{cal}}{\def\sol{\textast{0}}
%  \def\mercurius{\textast{1}}
%\astroversion{cal}  %Default will be `cal'

\thispagestyle{empty}

\begin{document}
%\tableofcontents
\noindent
\begin{center}

\normalsize
\textbf  {ABHANDLUNGEN AUS DER HAMBURGER STERNWARTE}\\
\textbf  {BAND XII, Teil 1}

\vspace{4.5cm}

\huge
\textbf{CATALOGUE ~~OF~   GALACTIC \\
PLANETARY ~NEBULAE }

\vspace{0.5cm}
\large

                 \textbf{UPDATED VERSION 2000}\\

\Large
\vspace{1.2cm}
                 \textbf{PART 1  (CATALOGUE)}

\vspace{1.8cm}


                 \textbf{ L. Kohoutek}\\


\vspace{1.cm}
\small
                 \textbf{technical collaboration}\\
                 \textbf{ D. K�hl}\\ 
\vspace{0.5cm}
                 \textbf{Hamburger Sternwarte}\\

\vspace{7.0cm}
\normalsize
                 \textbf{H A M B U R G - B E R G E D O R F~~   2 0 0 1}
\end{center}

\newpage
\thispagestyle{empty}
\vspace*{20cm}
%\bxtnorm
\noindent
ISSN 0374-1583\\
Copyright 2001 by Hamburger Sternwarte\\
All rights reserved\\
Print: Universit\"at Hamburg\\


\newpage
\vspace*{3.cm}
\begin{center}
To my respected teacher Dr. Lubo\v{s} Perek.
\end{center}
\newpage

%%_________________________________________________________________


\newpage
\section*{\textbf{C O N T E N T S :}}

\begin{tabular}{lr} 
							 &      \\
                                                         &      \\
\multicolumn{2}{l}{\bf P A R T  ~~1~~   ( C a t a l o g u e )}          \\
							 &      \\
Abstract \hfill ............................................................................&5\\
							 &      \\
1. Introduction \hfill .................................................................&    6\\
2. Scope of the Catalogue and General Remarks \hfill..........................          &    7\\
3. Objects Included and Omitted \hfill........................................&  11   \\
4. Explanation of Tables \hfill...............................................&  13    \\
5. Finding Charts \hfill......................................................&  16    \\
6. Elementary Statistics \hfill...............................................&  17    \\
Acknowledgements \hfill.......................................................&  36    \\
                                                         &      \\
Errata to CGPN and to the Supplements \hfill..................................&  38    \\
                                                         &      \\
References \hfill.............................................................&  41    \\
		
    					                 &      \\
Table 1. General List of PNe and  Misclassified PNe      &      \\          
\hspace{3.1cm}      ~ (according to galactic longitude) \hfill....................&  43    \\
							 &      \\
Table 2. List of PNe~ (according to right ascension) \hfill....................&  97    \\
							 &      \\
Table 3. List of Misclassified PNe  \hfill.....................................& 135     \\
                                                         &      \\
Table 4. Accurate Coordinates of PNe  \hfill...................................& 143     \\
							 &      \\
Table 5. Possible pre-PNe  \hfill..............................................& 265     \\
							 &      \\
Table 6. Possible post-PNe  \hfill.............................................& 281     \\
                                                         &      \\
Table 7. Index of Discovery Lists  \hfill......................................& 287     \\
							 &      \\
                                                         &      \\
\multicolumn{2}{l}{\bf P A R T  ~~2~~   ( C h a r t s )}	        \\
							 &      \\
Table 8. List of Finding Charts  \hfill........................................& 295     \\
							 &      \\
Finding Charts (according to galactic length) \hfill..........................& Plate 1 - 119      \\            
\end{tabular}

\newpage

\begin{center}
\section*{\textbf{ABSTRACT:}}
\end{center}


This catalogue contains 1510 objects classified as galactic PNe
up to the end of 1999. \mbox{CGPN(2000)} is a continuation of CGPN(1967)
and also includes objects given in Supplements S1-S6. The catalogue is divided into two parts:\\

\noindent
Part 1 - text and lists of objects (with corresponding remarks)
         given in the following 
\hspace*{1.7cm}      tables:\\
\hspace*{0.8cm}   Table 1: general list of PNe according to galactic longitude
            (including also misclassified 
\hspace*{1.7cm}      PNe),\\
\hspace*{0.8cm}   Table 2: list of PNe according to right ascension,\\
\hspace*{0.8cm}   Table 3: list of misclassified PNe,\\
\hspace*{0.8cm}   Table 4: accurate coordinates of PNe,\\
\hspace*{0.8cm}   Table 5: list of possible pre-PNe,\\
\hspace*{0.8cm}   Table 6: list of possible post-PNe,\\
\hspace*{0.8cm}   Table 7: index of discovery lists;\\

\noindent
Part 2 - finding charts of objects listed in S1-S6 and of some
         objects from CGPN(1967) 
\hspace*{1.6cm}     (not the charts given correctly in CGPN(1967));\\
\hspace*{0.8cm}   Table 8: list of finding charts.\\

The PK designation as well as the IAU PN G designation is indicated (Table 1 and 2), and equatorial coordinates for both the
Equinox 1950 and 2000 are presented (Table 2 and 4). Two new lists are given: the list of possible pre-PNe (Table 5, n=334), and the list of possible post-PNe (Table 6, n=86).\\

In the elementary statistics of the present data of PNe we give the galactic distribution as well as the distribution on the sphere, the occurence of coordinates in different accuracy categories, the existence of finding charts and also the frequency of discoveries.\\

We also present some summarized lists of PNe: individual distances, binary and variable objects, the occurrence in clusters and x-ray sources. The summary of misclassified PNe is added.\\

The errata to CGPN(1967) and to the Supplements are summarized.\\

\newpage

\begin{center}
\section{\textbf{Introduction}}
\end{center}

\noindent
This is the updated version of the ``Catalogue of Galactic Planetary 
Nebulae'' (Perek, Kohoutek, 1967) and includes the objects
classified as PNe or possible PNe until 1965 and presented in CGPN(1967), as well as in the following supplements: \\
\hspace*{0.8cm}   S1 - new objects published in 1966-1977,\\
\hspace*{0.8cm}   S2 - 1978-1981,\\
\hspace*{0.8cm}   S3 - 1982-1986,\\
\hspace*{0.8cm}   S4 - 1987-1990,\\
\hspace*{0.8cm}   S5 - 1991-1994,\\
\hspace*{0.8cm}   S6 - 1995-1999.\\
Supplements S1-S5 to CGPN(1967) have been published (Kohoutek 1978, 1983, 1989, 1992, 1997), whereas 
the last Supplement 6 is only available from the author.\\

This version CGPN(2000) is a continuation of CGPN(1967). Moreover it has two new lists which appeared in 
Supplements 4, 5 and 6 only: the list of pre-PNe and the list of post-PNe. They deal with objects which 
have not been classified as regular PNe, but which are as we believe in the evolutionary stage before PNe 
(pre-PNe) and after PNe (post-PNe), respectively. These lists are incomplete and reflect the opinion of 
their authors.\\

We still use equinox 1950.0 of coordinates in all tables for various reasons. But rough coordinates of all 
PNe given in equinox 2000.0 are also listed (Table 2).\\

As in CGPN(1967) the definition of a planetary nebula has
been taken in a rather wide sense. The total number of PNe is at present 1510 (classified till the end 
of 1999): 1183 objects classified more or less reliably as planetary nebulae; 
in addition 327 possible planetaries (denoted in the tables with an * ). Meanwhile 245 objects have been 
removed for different reasons from the CGPN(1967) and from the Supplements as misclassified PNe; there 
certainly exist also objects in the present version of the Catalogue which do not belong here. 
The definitive classification of the objects is not at all simple.\\

 Planetary nebulae are presented in this catalogue independently of the fact whether the objects are 
confirmed PNe or possible PNe (asterisk behind the designation), which fact may change already in the 
near future.\\

There are several objects called proto-planetary nebulae (PPNe). One of the first lists
of such objects was compiled and published in S1. The particular list of PPNe does not
exist in this catalogue: PPNe can be found either in the main list of PNe (Table 1), or
in the list of possible pre-PNe (Table 5). The separation of these two categories is
not reliably defined.\\

CGPN(1967) included all observational data on the objects. The
amount of such data increased to such an extent in the last thirty years that we found it necessary to 
restrict ourselves only to the data belonging to the location and identification of the objects, i.e. to 
coordinates as well as to finding charts. For the remaining data one may refer to \mbox{SIMBAD} 
(CDS, Strasbourg, France) as well as to the literature concerning the individual objects.\\

The Hubble Space Telescope is of great importance for the study of PNe, particularly concerning the 
morphology of PNe and the binarity of their nuclei. There are several other satellites which contributed 
essentially to the increase of our knowledge of PNe, especially HIPPARCOS, IUE, IRAS, ISO and ROSAT. Although 
the contribution of telescopes on satellites is increasing very much and investigation of PNe in certain 
spectral regions is only possible from beyond the Earth's atmosphere, the majority of the data on PNe is 
at present still produced at observatories on the ground.\\

Unfortunately the AAO/UKST H$\alpha$ Survey (Parker et al., 1999 and earlier announcements) could not be 
included in this updated version. This extensive 3-hour H-alpha photographic survey was started in 1997, 
should be completed in 2000/2001, and it is expected that >1000 new PNe will be found.\\

The structure of larger objects (>4 arcsec) in the northern sky can be seen in The IAC Morphological Catalog 
of Northern Planetary Nebulae (Manchado et al., 1996), which shows the sometimes complicated morphology of 
the objects called planetary nebulae. The structure of very large objects (>8 arcmin) is visible in the 
Atlas of Ancient Planetary Nebulae (Tweedy, Kwitter, 1996).\\

Valuable information about the objects including their identification charts can be found in the 
Strasbourg-ESO Catalogue of Galactic Planetary Nebulae (Acker et~al.,1992-\mbox{SECGPN}) and in its First 
Supplement (Acker et~al.,1996-S1 to SECGPN), where also the lists of publications are given. The 
classification criteria are somewhat different in SECGPN compared with CGPN, so that the objects
included in both catalogues slightly differ. \\

This updated version does not contain identification charts of all objects. We do not repeat those charts 
which have already been published in CGPN(1967), except some charts having wrong or uncertain identification. 
Concerning the correct charts, which is the large majority, the user of the present version is referred to 
CGPN(1967). In CGPN(2000) the identification charts are mostly of objects discovered after 1965 which were 
included in the Supplements.\\

The idea of publishing the updated version of CGPN arose in the
year 1976, and this was mentioned in Supplement 1. We remind the reader that the publication of CGPN was 
discussed already at the meeting of Commission 34 at the XII General Assembly of the I.A.U. in Hamburg, 
August 1964. This has been described in more detail in CGPN(1967).\\ \\

\begin{center}
\section{\textbf{Scope of the Catalogue and General Remarks}}
\end{center}

\noindent
In order to answer the fundamental question

\hspace{2cm}\textbf{``~what is a planetary nebula?~''}\\ 
we have compared the properties of various objects considering the current review literature and came to the following conclusion: 
A planetary nebula is a mainly gaseous object (also containing
dust) expanding from its hot central star of intermediate mass in a late evolutionary phase on the way between red giants and white dwarfs. The central stars ionize and illuminate the respective nebulae.  Although the parameters of the nebulae change very much during the rapid evolution of their nuclei and are therefore dependent on age, it is possible to summarize their typical values (extreme values are given in parenthesis).\\

\noindent
\textbf{N e b u l a :}\\
\hspace*{0.8cm}   Morphology: objects of mostly symmetrical shape (circular or elliptical discs or rings,
\hspace*{1.6cm}        sometimes bipolar structure with ``equatorial'' torus and ``pole'' condensations),
\hspace*{1.6cm}        with apparently sharp outer boundaries; often multiple shells (main nebula + faint  
\hspace*{1.6cm}        outer structure or halo). The morphology depends on the wavelength (stratifica-
\hspace*{1.6cm}        tion); it also reflects the intrinsic absorption and the orientation in space.\\
\hspace*{1.6cm}        Some objects have envelopes of neutral hydrogen and molecules.\\
\hspace*{0.8cm}   Angular diameter: $\thicksim$ 20" - 40", depending on the wavelength  (limits stellar - $\thicksim$ 20', 
\hspace*{1.6cm}        Sh 2-216 even larger).\\  
\hspace*{0.8cm}   Spectrum:\\
\hspace*{1.6cm}      emission lines:\\
\hspace*{2.4cm}         (a) recombination lines mostly of H and He;\\
\hspace*{2.4cm}         (b) collisionally excited (forbidden) lines of C, N, O,  Ne, Mg, Si, S, Cl, Ar; \\
\hspace*{2.4cm}         (c) fluorescent lines (rare) of OIII and NIII.\\
\hspace*{1.6cm}      continuum emission:\\
\hspace*{2.4cm}         free-bound, free-free, two-quantum processes, emission from grains (dust).\\
\hspace*{1.6cm}      Spectrum depending on excitation conditions (exc. class), stratification, chemi-\\
\hspace*{2.4cm}         cal composition.\\
\hspace*{1.6cm}      Exc. classes: 0 - 10, main criteria: I([OIII]$\lambda$5007 $+ \lambda$4959)/I(H$\beta$), \\
\hspace*{8.0cm}                       I(HeII$\lambda$4686)/I(H$\beta$), \\
\hspace*{8.0cm}                       I([OII]$\lambda$3727)/I([OIII])$\lambda$4959). \\
\hspace*{2.4cm}         (see also below)\\
\hspace*{0.8cm}   IR-spectrum: nebular emission lines, dust continuum, IR-emission features.\\
\hspace*{0.8cm}   IR-fluxes: \mbox{F$_{\nu}$(12$\mu$m)/F$_{\nu}$}(25$\mu$m) $\leqq$ 0.35, F$_{\nu}$(25$\mu$m)/F$_{\nu}$(60$\mu$m) $\geqq$ 0.3,\\
\hspace*{2.4cm}         H2 (some objects). \\
\hspace*{0.8cm}   Radio emission: continuum, mainly molecules CO, OH.\\
\hspace*{0.8cm}   Dimension: diameter 0.1 pc - 0.2 pc (limits $\thicksim$ 0.005 pc or even smaller, $\thicksim$ 7 pc) 
\hspace*{1.6cm}       depending on wavelength.\\
\hspace*{0.8cm}   Electron density: 10$^{3}$ - 10$^{4}$cm$^{-3}$ (but also <10$^{3}$cm$^{-3}$ for old objects and >10$^{4}$cm$^{-3}$ 
\hspace*{1.6cm}      for young objects possible).\\
\hspace*{0.8cm}   Electron temperature: 9 000$^\circ$K - 15 000$^\circ$K (limits 8 000$^\circ$K -  23 000$^\circ$ K).\\
\hspace*{0.8cm}   Total mass: ionized gas: 0.1 M$_{\odot}$ - 0.2 M$_{\odot}$ (limits $\thicksim$ 0.001 M$_{\odot}$, $\thicksim$ 1 M$_{\odot}$);\\
\hspace*{3.0cm}            neutral gas~+~dust: sometimes much higher, very different;\\ 
\hspace*{3.0cm}      dust mass/gas mass $\thicksim$ 2x10$^{-4}$ to 3x10$^{-2}$.\\
\hspace*{0.8cm}   Expansion velocity: non-isotropic, $\thicksim$ 25 km/s (limits 4 km/s, 60 km/s, outer con- 
\hspace*{1.6cm}     densations up to $\thicksim$ 300 km/s).\\
\hspace*{0.8cm}   Age: 0 - $\thicksim$ 100 000 years.\\

\noindent
\textbf{N u c l e u s :}\\
\hspace*{0.8cm}   Spectrum: WR, O, Of, WR+Of, OVI, sdO, cont., peculiar, sometimes variable.\\
\hspace*{0.8cm}   Effective temperature: 40 000$^\circ$K - 100 000$^\circ$K (limits $\thicksim$ 20 000$^\circ$K - $\thicksim$ 250 000$^\circ$K). \\ 
\hspace*{0.8cm}   Luminosity: $\thicksim$ 5x10$^{3}$L$_{\odot}$ (limits $\thicksim$ 10 L$_{\odot}$, $\thicksim$ 10$^{4}$L$_{\odot}$).\\
\hspace*{0.8cm}   Radius: limits $\thicksim$ 0.005R$_{\odot}$, $\thicksim$ 1.5R$_{\odot}$.\\
\hspace*{0.8cm}   Mass: $\thicksim$ 0.6M$_{\odot}$ (progenitors between 0.8M$_{\odot}$ and 6-8M$_{\odot}$).\\
\hspace*{0.8cm}   Mass loss: var $\thicksim$ 10$^{-10}$M$_{\odot}$/yr - 10$^{-7}$M$_{\odot}$/yr ($\thicksim$ 10$^{-5}$M$_{\odot}$/yr in late AGB).\\
\hspace*{0.8cm}   Gravity: log g $\thicksim$ 3.0 - 7.5.  \\                

The mean parameters of the nebulae and of their nuclei have been taken over mainly from Supplement 3 (with the
extension given in Supplement 5). The summary of the criteria by which PNe are distinguished from several types of objects is given in Supplement 2.\\

\textbf{Evolution of PNe}\\

The main group of PNe is part of the so-called "blue-white sequence" which was already introduced by Vorontsov-Velyaminov (1947) for the description of the positions of PN nuclei in the HR diagram. This sequence was explained as the place of occurence of planetary nuclei in their evolution between red giants and white dwarfs (Vorontsov-Velyaminov, 1948 and 1953; \mbox{Shklovsky,} 1956; Harman, Seaton, 1964; and later). The actual best explanation of the expansion of the nebular envelopes was found in the model of the Interacting Stellar Winds (Kwok et al., 1978; Kwok 1982) based on the mass loss coming from AGB stars. The evolutionary track of central stars was calculated mainly by Paczy\'nski (1971), Sch\"onberner (1979) and Bl\"ocker (1995).\\

 There are two periods in the evolution of PNe which are still vaguely understood at present:\\

(a) The first period concerns the  o r i g i n  of PNe and the evolution of the new-born PN. It is generally believed, and the
theoretical evolutionary tracks support this, that PNe evolve
from red giants to post-AGB objects via stellar winds. Especially IRAS 
objects with colours similar to those of common PNe are
highly suspected of accomplishing this evolution and they therefore deserve
 our particular attention. Nevertheless such objects are not yet classified generally as PNe; they are listed as~ \textbf{p r e - PNe} in a separate table (Table 5). Exceptionally some of these
objects are classified as PNe even if the well-known nebular lines [OIII]$\lambda$5007,$\lambda$4959 are missing; in these cases other classification criteria support the idea that they are after all early PNe. \\

(b) The second period concerns very old PNe, where the nebulae
have already disappeared and the central stars are very faint and similar to common WD. It is very difficult to detect such objects with the present observational techniques. Again in a separate table (Table 6) we list these objects as~ 
\textbf{p o s t - PNe.}\\

\textbf{Excitation classes}\\

The \mbox{exc.~classes} were introduced in order to classify the
spectra of planetary nebulae using the level of excitation
(excitation potential of the emission lines). The classification
criteria are based on the scheme which was developed mainly by
Aller (1956), partly already by Page (1942) or even earlier.
Only one criterion is sensitive over nearly the whole range of 
excitation: I([OIII]$\lambda$5007+$\lambda$4959)/I(H$\beta$). Moreover the
ratio I(HeII $\lambda$4686)/I(H$\beta$) is suitable for high and very high,
 whereas I([OII]$\lambda$3727)/I([OIII]$\lambda$4959) for low 
and medium excitation classes. Unfortunately this last criterion should be taken
with caution because of possible strong interstellar absorption which might
weaken the $\lambda$3727 line.
For low excitation classes the ratio 
I([NII]$\lambda$6584)/I(H$\alpha$) is also useful. We added \mbox{exc.~classes}
0-1 and 1 for very-low-excitation (VLE-objects were introduced 
already by Sanduleak, Stephenson, 1973) mostly compact objects:
in
\textbf{\mbox{exc.~class 1}} [OIII]$\lambda$5007 is very weak but
still visible, 
\textbf{\mbox{exc.~class 0 - 1}} is reserved for objects showing no visible 
[OIII]$\lambda$5007 line. In this case some additional indications 
for PNe (e.g. non-stellar angular diameter, continuous spectrum,
infrared fluxes) should be given. The above exc. classes contain mainly
objects the star temperatures of which are too low for producing the
N1 line, but which are on the evolutionary way to common PNe. 

The proposed scheme of excitation classes 0-10 is as follows:
\\

\begin{tabular}{ll}
exc. class   & exc. level	\\
\hline
0-1 and 1    & very low		\\
2-3          & low		\\
4-5          & medium 		\\
6-7          & high		\\
8,9 and 10   & very high	\\
\end{tabular}
\\\\

\textbf{Designation}\\

We intentionally use the same designation of PNe as in the edition CGPN(1967), i.e. in the system: \textbf{lll $\pm$  bb.n, where n=1,2 \dots} is the number of the object in the respective area 1$^\circ$ x 1$^\circ$. This designation (and not the more detailed one) was used in CGPN(1967) also in order to avoid a possible confusion due to the sometimes approximate coordinates (the positional accuracy of 26 \% of the objects in CGPN(1967) was not better than 1 arcmin). At present the positional accuracy has been improved, and the designation can therefore be more detailed. We also give (Table 2) the IAU designations of galactic planetary nebulae, PN G lll.l $\pm$ bb.b, recommended by IAU Commission 5 (Astronomical Nomenclature) and also used in SECGPN. In order to avoid possible confusion we use the same PN G designation of objects as given in SECGPN; this is also in case our galactic coordinates would differ slightly from those of SECGPN due to improved
equatorial coordinates. \\

We are of the opinion that the confusion concerning the PK designation which sometimes occurred in the literature was avoidable. We would have expected that the discoverer would publish the new objects either together with their galactic coordinates only, or with the first part of the designation, containing the galactic longitude and latitude and  n o t  with the number of the object in the respective area 1$^\circ$ x 1$^\circ$ (e.g. 255$-$15. only). If necessary it would have been possible to distinguish several objects in one area of the galactic longitude and latitude with letters A, B,..  (e.g. 255$-$15.A, 255$-$15.B,...). The definitive numbering of the objects in this area should have been reserved for the person who wrote the respective supplement to CGPN or the new updated version of this catalogue. As an example: there are several PK designations in the new catalogue of symbiotic stars (Belczy\'nski et al., 2000, Table 8), which did not appear either in CGPN(1967) or in Supplements S1-S5 and which are not correct. To use them could therefore be confusing. - 
The above procedure has not been explicitely mentioned in CGPN(1967), but it was assumed to be self-evident; in reality it was sometimes not. For this reason we shall write about this matter in more detail now. 


\begin{center}
\section{\textbf{Objects Included and Omitted}}
\end{center}

\protect \vspace{0.3cm}
\noindent
We believe all objects classified as PNe and published until the end of 1999 are included in this version CGPN(2000). It contains PNe from CGPN(1967) as well as PNe discovered later and published in the supplements. These objects appear in  \textbf{Table 2: List of PNe} (n=1510). Objects which have been removed from the list of PNe are given in \textbf{ Table 3: List of misclassified PNe} (n=245). Objects from both tables are included in a \textbf{general list of PNe (Table 1)} (n=1755): all objects once classified as PNe, having therefore the PK designation, i. e.

          n(Table 1) = n(Table 2) + n(Table 3).\\
Exceptions are Cn 1-1, MaC 1-1, PC 11 and M 1-27, which were once removed from PNe but later included again in this list.\\

We include in this catalogue only objects classified as PNe and given in \textbf{published} lists (at least in lists which appear in preprints) and not just given in lists in papers which are ``~in preparation~''.\\

At present we include in the list of PNe several objects which
also appear in the lists of symbiotic stars (SS). In our opinion it is
not advisable to classify the objects\textbf{ either} as \textbf{SS or} as 
\textbf{PNe},
and therefore some of them can appear \textbf{simultaneously} in both lists. We point out that the hot components of symbiotic stars resemble the central stars of PNe (luminosity, temperature, mass, diameter), being also on the post-AGB tracks, which means in the region of the HR diagram occupied by PNe. We would also like to emphasize the existence of the bipolar nebulae He 2-104 (Southern Crab, Schwarz et al., 1989) and BI Cru (Schwarz, Corradi, 1992), the classification of which as PNe seems to be probable already from their morphology, although these objects were included in the main list of SS (Allen, 1984). Allen's list of SS contains 144 objects and 44 of them are simultaneously classified as PNe! Also the new catalogue of SS (Belcz\'ynski et al., 2000) contains 53 objects (24\% of known SS), which are simultaneously classified as PNe.
It seems that bipolarity as a result of an anisotropic mass-outflow is a phenomenon common for young PNe as well as for some SS, so that there exist emission-line objects having high-velocity bipolar nebulae and high core densities, which can be called ``\textbf{symbiotic proto-PNe}''. Probably there is a link between bipolar PNe and SS (Corradi, 1995). Also the observed continuum in the IR and the presence of TiO bands, which give evidence for a late-type (G-K-M or carbon) star and which is a strong criterium for SS, cannot be a reason for excluding such objects from the list of PNe, because PNe having binary (symbiotic) central stars may also exist. We emphasize the binary system BE UMa which consists of the components sdO+K5. Incidentally there exists a small group of so-called ``yellow'' symbiotic stars, named so because their cool components are of spectral type of about F. As noted in \S 6 (Elementary Statistics) the percentage of known binary and variable stars among the PN central stars is noticeably smaller compared with that of the ``common'' stars, so that a large number of binaries have probably not yet been detected. As for visual binaries a considerable contribution was made
by using the Hubble Space Telescope (Ciardullo et al., 1999); the question
of their physical association still remains open in many cases.\\

Our \textbf{criteria for excluding objects from the list of PNe as SS} are more narrow than the conventional definition of SS is. We
exclude only such objects from the list of PNe which show\\
\hspace*{0.8cm}   1) strong HeII$\lambda$4686 emission line and simultaneously no (or nearly no) [OIII]$\lambda$5007, \mbox{$\lambda$4959 lines,} and\\
\hspace*{0.8cm}   2) the presence of a 6825A emission feature, as in  
M 1-21 (006$+$07.1), He 2-417 \mbox{(012$-$07.1)} and He 2-374 (009$-$02.1).\\

Both criteria are empirical and reflect the appearance of some bright emission features in the optical region (visible also for faint objects). As to absorption features and the presence of the continuum of a late-type star, these are according to our opinion not sufficient criteria which distinguish necessarily between the SS and the (binary) central stars of PNe.\\
\hspace*{0.6cm}ad 1) The appearance of both bright HeII$\lambda$4686 and N1, N2 lines is evidence for a PN of high or very high excitation class. But if we observe very weak N1, N2 and simultaneously bright HeII$\lambda$4686 lines it would be strange for a PN even if we take into account different excitation zones.\\
\hspace*{0.6cm}ad 2) The appearance of the emission bands at $\lambda$$\lambda$6825A, 7088A, for a long time regarded as unidentified and even a little mysterious, is explained as a Raman scattering by neutral hydrogen of the OVI doublet $\lambda$$\lambda$1032A, 1038A (Schmid, 1989). It is proposed that OVI photons from a hot radiation source are absorbed by hydrogen near the cool giant atmosphere. These bands are observed only in the spectra of SS and not of PNe.\\  

To find the spectral properties distinguishing between PNe and
SS is of course useful, because there evidently exist some differences between them. We believe that the largest spectral differences between SS and PNe are given in the criteria presented above.\\

As in CGPN(1967) we left in the list of PNe several doubtful objects and objects having no emission lines detected at 
present. We have to accept the fact that some of them will be removed later from the list of PNe.\\

As written already in the Introduction we also present two lists
of objects, which have not been classified as regular PNe, but
which are in their evolution close to them: \textbf{\mbox{pre-PNe} \mbox{(Table 5)}},
objects being between post-AGB and very young PNe, and \mbox{\textbf{post-PNe}}
\mbox{\textbf{(Table 6)}}, where the nebulae in most cases have already disappeared and the
central stars are on the evolutionary way to WD.\\

There exists moreover a large group of objects which can be
called \textbf{PN candidates}. These are partly objects from internal
(unpublished) lists, but mainly objects from already published lists, which have not sufficient data to be classified as PNe. Before further observations and their interpretation are available, such objects can only be considered as PN candidates. In the past the objects were sometimes included in the list of PNe rather carelessly and without further confirmation. Also this fact led to a relatively large number of misclassified PNe. The classification as a PN should be made more thoroughly in future, and the category PN candidates should be reserved for the objects having insufficient data.\\

This also concerns distant objects in regions of large interstellar extinction, which is in the galactic equator, especially in the direction towards the galactic centre. Such objects are not visible in the optical region; only measurements made in IR- and radio-regions are at our disposal. This means that the common criteria for PNe (mainly the presence of the emission lines [OIII]$\lambda$5007,$\lambda$4959 or [NII] near H$\alpha$) cannot be applied.\\

Pottasch and collaborators (Pottasch et al., 1988) found that PNe appear in a certain region of the colour-colour diagram for IRAS sources. On the other hand, IRAS objects from the same region of this diagram (i.e. with far-IR colours that are typical for PNe) may be considered as PN candidates. Preite-Martinez (1988) using very similar criteria gave a classification of 388 IRAS sources from the IRAS PSC as possible new PNe. For some PN candidates radio measurements of continuum emission were made which show the presence of ionized gas. The detection of radio continuum emission provides strong evidence that the objects may indeed be PNe. However even this fact is at present not sufficient for classifying such objects as PNe.\\

Helfand and his group (Becker et al., 1994; Kistiakowsky, Helfand, 1995) classified many objects as PN candidates using VLA. Sometimes nothing at all is visible at the position of such PN candidates in the optical region, but many candidates have a bright 
line [SIII]$\lambda$9532 stronger than [OIII]$\lambda$5007.\\

Let us remind the reader that so far all known PNe have been recognized optically. Nevertheless it may be that in future many new additions to the group of PNe will be made as a result of radio and infrared studies only. For such a purpose clear criteria for classifying the objects as PNe according to their properties in IR and radio regions should be postulated. At present such objects were not included in this catalogue.\\ \\





\begin{center}
\section{\textbf{Explanation of Tables}}
\end{center}

\noindent
\hspace*{0.8cm}\textbf{   Table 1: General list of PNe together with misclassified objects} (according to galactic 
longitude)\\
\hspace*{1.6cm}       PK designation: designation as in CGPN(1967) (see \S 2);                         \\
\hspace*{1.6cm}       Name: name of the object as PN (see Table 7: Discovery Lists);                   \\
\hspace*{1.6cm}       RA, DEC: rough coordinates (equinox 1950.0) having the accuracy 0.1$^{m}$, 1';   \\
\hspace*{1.6cm}       Discovery: discoverer of the object \textbf{as PN} (see Table 7: Discovery Lists) taken\\
\hspace*{2.4cm}                from CGPN(1967) (blank) or from S1-S6; discoverer of misclassified object\\
\hspace*{2.4cm}                (marked M) is given in parentheses; the supplement where the object was \\
\hspace*{2.4cm}                misclassified is also indicated.                         \\
\hspace*{1.6cm}       Remarks: mainly independent discoveries, classif., identification, other names.\\                                      

This \textbf{general list} contains not only real PNe (objects from Table
2), but also objects which were once classified as PNe but later
removed from that list (objects from Table 3). The existing PK designation has been reserved for the given object only and is independent of its possible later new classification.\\
       
\noindent
\hspace*{0.8cm}\textbf{   Table 2: List of PNe} (according to right ascension)\\
\hspace*{1.6cm}       PK designation  (same as in Table 1);\\
\hspace*{1.6cm}       Name (same as in Table 1);\\
\hspace*{1.6cm}       RA, DEC: rough coordinates (equinox 1950.0) having an accuracy 0.1$^{m}$, 1';\\
\hspace*{1.6cm}       Finding Charts: Plates Nos.1-119 (all objects given in S1-S6 and corrected charts 
\hspace*{2.4cm}           of CGPN(1967));\\
\hspace*{1.6cm}       RA, DEC: rough coordinates (equinox 2000.0) having an accuracy 0.1$^{m}$, 1';\\
\hspace*{1.6cm}       PN G designation: taken from SECGPN or S1,\\
\hspace*{2.4cm}           also possible PNe and rejected objects from SECGPN, ---not present objects; \\
\hspace*{1.6cm}       Remarks: mainly different names for objects given in SECGPN or S1.\\

This is the \textbf{actual list of PNe (till 1999)}. The name of the nebula is in most cases identical with the abbreviation of the
person(s) who classified the object as a planetary nebula, or
who gave the nebula his(their) own abbreviation. (For the abbreviations see Table 7: Index of discovery lists.) The exceptions
are given in the CGPN(1967) or in the supplements.\\

There are 327 objects classified at present as ``possible planetary nebulae'' and denoted by  * , added to the PK designation.
These objects especially require further confirmation, which does not mean further observational data only, but also more precise 
theoretical consideration about what a planetary nebula is.\\

In the last column of this table and in the remarks the comparison with SECGPN is given. There 
are 63 objects included in this catalogue, which were rejected from SECGPN, and 101 objects which were classified there as possible planetaries. The objects rejected from 
SECGPN were mainly classified as symbiotic stars (n=35), peculiar em.-line stars (n=9)
or they are objects where no em.-lines were observed (n=12). Our criteria for excluding objects from the list of PNe as SS are described in the former paragraph. As to the
objects showing no em.-lines we are of the opinion that this situation is temporary only and that it may in principle change in the near future when better observational techniques will be used. On the contrary there are over 50 objects included in SECGPN and its S1 and not appearing in the main list of PNe in this catalogue. You can find a part of them as possible pre-PNe (Table 5) and possible post-PNe (Table 6). In general the difference between CGPN(2000) and SECGPN can be explained by the somewhat different
classification criteria for PNe.\\

\noindent
\hspace*{0.8cm}\textbf{   Table 3: List of misclassified PNe} (according to right ascension)\\
\hspace*{1.6cm}      PK designation (same as in Table 1);\\
\hspace*{1.6cm}      Name (same as in Table 1);\\
\hspace*{1.6cm}      RA, DEC: rough coordinates (equinox 1950.0) having an accuracy 0.1$^{m}$, 1';\\
\hspace*{1.6cm}      Misclassified: in S1-S6;\\
\hspace*{1.6cm}      Remarks: why the objects were misclassified.\\

This table contains objects once classified as PNe but removed
from the list of PNe in S1-S6. In the remarks only rough reasons
for misclassification are given; for more details see the corresponding supplement and the literature given there.\\

\noindent
\hspace*{0.8cm}\textbf{   Table 4: Accurate coordinates of PNe} (according to right ascension)\\
\hspace*{1.6cm}      PK designation (same as in Table 1);\\
\hspace*{1.6cm}      Name (same as in Table 1);\\
\hspace*{1.6cm}      RA, DEC: accurate coordinates (for the given equinox) having an accuracy\\
\hspace*{2.4cm}               (a) 0.01$^{s}$, 0.1"\\
\hspace*{2.4cm}               (b) 0.1$^{s}$, 1" \\
\hspace*{2.4cm}               (c) 1$^{s}$, 0.1'(mostly large objects);\\               
\hspace*{1.6cm}      Source: reference to the published literature or to the star catalogues;\\
\hspace*{1.6cm}      Remarks: mainly concerning the individual coordinates and our measurements.\\

Many coordinates from the literature having accuracies (a) and (b) are listed in this table. For some objects only coordinates 
with the accuracy (c) were found in the literature; it is mainly a question of large
 nebulae without known and visible central stars. For about 800 objects the coordinates were improved (Kohoutek, K\"uhl, 1999) and measured on the Digitized Sky Survey with the accuracy (b) or exceptionally (c) for large objects. Let us mention that
the coordinates given in various papers need not be independent
of each other - the nature of the published coordinates was
not indicated in all papers. Our aim was of course to use the
original coordinates only.\\

Sometimes there are discrepancies in the individual coordinates.
Larger discrepancies and probably typing errors are mentioned in
the corresponding remarks.\\

The sources of coordinates can be divided into the following groups:\\ 
\hspace*{0.7cm}(a)~ coordinates given already in CGPN(1967) (34 entries);\\
\hspace*{0.7cm}(b) coordinates from various papers 1969-1999 reported in Astronomy Astrophysics 
\hspace*{1.3cm}            Abstracts (volume, paragraph, number); there is one paper which appeared bet-\\
\hspace*{1.3cm}            ween 1967 and 1969 and
which was reported in AJB (Astronomisches Jahrbuch);\\
\hspace*{0.7cm}(c) coordinates given in the main catalogues (AC, AGK2, AGK3, GSC, HIPPARCOS, 
\hspace*{1.3cm}            IRAS, LS, LSS, PPM, SAO).\\

As to the IRAS sources we distinguish three categories:\\
\hspace*{0.8cm}  (a) sources which are identical with the optical objects (the
difference between the optical and IRAS coordinates does not
exceed about 20") - they are given in the main table;\\
\hspace*{0.8cm}  (b) sources which are near the optical objects (the difference 20-60"). It is in principle possible that some of these IRAS sources are also identical with the optical objects, but the above difference exceeds the IRAS positional accuracy - such sources are given in the remarks;\\
\hspace*{0.8cm}  (c) sources more than 60" away from the optical objects are
not mentioned.\\ 
   
\noindent
\hspace*{0.8cm}\textbf{   Table 5: Possible pre-PNe} (according to right ascension)\\
\hspace*{1.6cm}      Name;\\
\hspace*{1.6cm}      RA, DEC: coordinates with various accuracies (equ. 1950.0);\\
\hspace*{1.6cm}      Remarks and references.\\ 

The list of possible pre-PNe (n=334) contains objects being on the evolutionary way between late
AGB stars and PNe. They are often IRAS sources with colours similar to PNe. Some of
these objects are also called proto-PNe. This list is incomplete and the references
give the orientation only; it does not appear in CGPN(1967).\\
   
\noindent
\hspace*{0.8cm}\textbf{   Table 6: Possible post-PNe} (according to right ascension)\\
\hspace*{1.6cm}      Name;\\
\hspace*{1.6cm}      RA, DEC: coordinates with various accuracies (equ. 1950.0);\\
\hspace*{1.6cm}      Remarks and references.\\ 

The list of possible post-PNe (n=86) contains objects the central stars of which are on the
evolutionary way between very old and large PNe and
white dwarfs. This list is incomplete and the references give the orientation only; 
it does not appear in CGPN(1967).\\

\noindent
\hspace*{0.8cm}\textbf{   Table 7: Index of discovery lists}\\ 

This list gives the abbreviations of the names of the discoverers used for names of planetary nebulae in this catalogue
(the catalogue CGPN(1967) as well as all supplements have been
included).\\

We repeat - and this is our main rule - that \textbf{that person
is regarded as discoverer, who first denoted the object as a
planetary nebula} (and not who discovered the object generally).\\

The new planetary nebula receives the PK designation (given by
the author of the catalogue), the IAU PN G designation (may be
given by the discoverer himself) and the name, which is either\\
\hspace*{0.8cm}   1) the common or generally used name of the object (NGC, IRAS, BD, ESO,...); or\\
\hspace*{0.8cm}   2) the abbreviation of the name of the discoverer (or first two discoverers maximum)
\hspace*{1.6cm}      according to Table 7, and given by the author of the catalogue; or\\
\hspace*{0.8cm}   3) the abbreviation or name proposed by the discoverer himself (sometimes the name 
\hspace*{1.6cm}      from his internal list). \\  

Our further main rule is that the \textbf{ discoverer receives only one
single abbreviation} which is the same in his different lists of
planetary nebulae published in different papers. (Unfortunately
there exist some exceptions mainly with regard to discoverer, to SECGPN or by error).
 Our procedure differs from that of SECGPN and its Supplement 1, where either different
abbreviations appear for one discoverer (example: Weinberger -
We, Wei, WKG - Weinberger, Kerber, Gr\"obner), or one abbreviation was given to different discoveres (example: St - Stephenson, 1978; StWr - Stock, Wroblewski, 1972; SwSt - Swings, Struve, 1940).\\



\begin{center}
\section{\textbf{Finding Charts}}
\end{center}

\protect \vspace{0.3cm}

\noindent
\hspace*{0.8cm}\textbf{Table 8: List of finding charts} (according to galactic longitude)\\
\hspace*{1.6cm}      PK designation (same as in Table 1);\\
\hspace*{1.6cm}      Name (same as in Table 1);\\
\hspace*{1.6cm}      RA, DEC: rough coordinates (equinox 1950.0) having an accuracy 0.1$^{m}$, 1';\\
\hspace*{1.6cm}      Discovery: CGPN (1967) or supplements S1-S6;\\
\hspace*{1.6cm}      Finding charts: FC - main charts, plates Nos. 1-119 (charts for all PNe given in  
\hspace*{5.5cm}          S1-S6 and corrected charts for objects from CGPN(1967));\\
\hspace*{4.8cm}                       C - central charts (crowded fields);\\
\hspace*{1.6cm}      Date-Obs: UT date of observation;\\
\hspace*{1.6cm}      P-Label: observatory plate label;\\
\hspace*{1.6cm}      Remarks: mainly concerning finding charts and identifications.\\

The finding charts are given if available for all PNe which
were discovered in S1-S6; we present also those charts from
CGPN(1967) which (a) are wrong, (b) have uncertain identification, or (c) show no surroundings (for some bright nebulae). For the remaining objects we refer to charts given in CGPN(1967). 
\\

The identification charts are normally oriented 
(\textbf{N at the top, W to the right}) and of size 
\textbf{10 arcmin x 10 arcmin} in general. For some large objects there are charts of 
 size 15 or 20 arcmin square. For objects in crowded fields we give also the central part of the corresponding charts having 2 arcmin square.
Such C-charts are on the same plates as the main charts and behind them. The C-charts have the same Data-Obs and P-Label as the main charts except in four cases: the corresponding deviations are indicated in Remarks.\\

Most objects are marked with a cross: star-like nebulae, small nebulae and central
stars of large nebulae, all measured by us (Kohoutek, K\"uhl, 1999), as well as nebulae
having accurate coordinates given in the literature. There are several PNe, the identification of which was made according to coordinates only. Such identification is of course uncertain. Nebulae without known central stars but well visible on finding charts
are marked with circles. If the abbreviation ``{\it ID}~'' between the PK designation and the
name of the nebula appears, there is no identification (n=35). These are mainly               objects not visible on our broad-band charts because the discovery pictures were taken through narrow filters only
either in the visual ([NII]$\lambda$6584, H$\alpha$, [OIII]$\lambda$5007) or in the near infrared ([SIII]$\lambda$9532). In these cases, or generally in all cases, in which the exact identification of the objects could not be given, the circles indicate 
again their position (the centres corresponding to the discovery coordinates), and we refer to the discovery literature.\\
            
The charts were taken from the Digitized Sky Survey and their colour system is given in Table 8. The \textbf{PK designation} and the \textbf{name} of the
respective object are given above each chart. If appropriate the central chart (C) and
no identification ({\it ID}) is also indicated.
For greater convenience the charts are supplied with the coordinate grid using the computing programme of H. Hagen (private communication), where the scale of the charts can also be seen.\\

The finding charts (main charts and central charts) are published in Part 2 of the catalogue with
their list given in Table 8. They are arranged according to the PK designation and correspond to Table 8. Their purpose is to give the base for identification of the objects and not for their morphology.\\ \\ 


\begin{center}
\section{\textbf{Elementary Statistics}}
\end{center}
\subsection*{Distribution of planetary nebulae}
\end{center}
The distribution of PNe in galactic coordinates is typical for objects of the disk population: high concentration towards the galactic centre and towards the galactic equator (Figs.1-4).\\
% Fig 1
\afterpage{\clearpage}
\begin{figure}[tp]
\rotatebox{-90}{\scalebox {0.95}{
\includegraphics*[bb= 45 115 335 580,clip]{hh0lb.ps}
}}
{\bf \mbox{Fig.1~~}}\mbox{The distribution of PNe in galactic coordinates.}

\end{figure}
\begin{figure}[bp]
\rotatebox{-90}{\scalebox {0.95}{
\includegraphics*[bb= 45 115 335 580,clip]{hh0ad.ps}
}}
{\bf \mbox{Fig.2~~}}\mbox{The distribution of PNe in equatorial coordinates.}

\end{figure}

%\newpage
\afterpage{\clearpage}
\begin{figure}[tp]
\includegraphics[bb= 100 355 570 720,clip]{hh0l1.ps}
{\bf \mbox{Fig.3a~~}}\mbox{Frequency of PNe in galactic longitude.}

\vspace{0.5cm}
\includegraphics[bb= 95 460 570 720,clip]{hh0l2.ps}
{\bf \mbox{Fig.3b~~}}\mbox{Frequency of PNe in galactic longitude in the
central region.}
\end{figure}

%\newpage
%\afterpage{\clearpage}

%\vspace*{-3.cm}
\hspace*{-0.5cm}
%\begin{picture}
\includegraphics[bb=100 380 560 720,clip]{hh0b1.ps}

\noident
{\bf \mbox{Fig.4~~}}\mbox{Frequency of PNe in galactic latitude.}\\ 


The distribution in galactic longitudes is more or less symmetrical (Fig.3a). But the more detailed histogram shows a somewhat lower frequency of objects between l 345${^\circ}$ and 355${^\circ}$ (Fig.3b). The concentration towards the galactic equator is very high (Fig.4) and 1275 PNe (84\%) are in the region b $\pm$ 10${^\circ}$.\\

The distribution on the celestial sphere shows again the presence of the galactic centre and the galactic equator (Fig.2). The appearance of PNe outside the centre is again not wholly symmetrical, objects between AR 12h-17h seem to be underpopulated compared with the region 19h-21h, which fact might be an observational effect.\\

\subsection*{Coordinates and finding charts}
\end{center}
The best coordinates for each object given in the literature and measured by us are listed in Table 4 (about 7000 entries from 218 sources). The frequency of coordinates in the accuracy categories is as follows:\\

\noindent
category (a): 964 objects\\
category (b): 544 objects\\ 
category (c):   2 objects\\
%\end{picture}
\afterpage{\clearpage}


Nearly all PNe (n=1508) have coordinates of the accuracy category (a) or (b). Only two objects of the category (c) are large nebulae without known central stars (VB 1, 339$-$00.1*; EL 1647+64, 094+38.1*). We give the individual coordinates and not the best coordinates only or their mean values in order to see the scatter of coordinates which sometimes is considerable, especially at large nebulae without known central stars.\\

The finding charts are given either in CGPN(1967) or in this Catalogue. In the present version there are 714 finding charts (655 main charts, 59 C-charts). In 35 cases the object is not visible (ID) and we refer to the discovery literature.\\
 
\subsection*{Discoveries}
\end{center}
The first object called planetary nebula was named more than two centuries ago (NGC 6720; Ch. Messier, A. Darquier, 1779). Due to the work of W. Herschel altogether 13 PNe were known at the end of the 18th century. In addition, further 65 PNe were found in the 19th century.\\



A substantial increase of new PNe can be registered since 1945 as a result of spectral surveys and of Schmidt cameras. (About 90\% of known PNe have been discovered since 1945.) In the last decade the slight increase of new PNe is mostly due to observations in no-visual regions (mainly in infrared) and due to the use of CCD cameras, which are
 more sensitive than photographic plates. The following catalogues or lists (partly mentioned already in CGPN(1967)) are given below:\\\\

\begin{figure}[t!]
\scalebox {0.93}{
\includegraphics*[bb=100 365 595 720,clip]{hh001.ps} }
{\bf \mbox{Fig.5~~}}\mbox{Dicoveries of PNe until 2000 in the respective decade.}
\end{figure}



\begin{tabular}{lrlc}
 Catalogue or list                   & \multicolumn{2}{c}{Number of}& Number of  \\
                                                        & \multicolumn{2}{c}{  PNe    }& miscl. PNe \\   
\hline						        	   	 
                                                       &	          &	&            \\
Curtis (1918)                                          &   ~~102	  &	&            \\
Vorontsov-Velyaminov, Parenago (1931)                  &   ~~121	  &	&            \\
Vorontsov-Velyaminov (1948)                            &   ~~288	  &	&            \\
Vorontsov-Velyaminov (1962)                            &   ~~591	  &	&            \\
Perek, Kohoutek (1967) - CGPN(1967)                    &   1036	  &	&            \\
~~~Kohoutek (1978-2000) - Supplements 1-6 to CGPN(1967)&   ~~721        &     &   247      \\
Acker et al. (1992) - SECGPN                           &   1143         &  *) &   330      \\
                                                       &   ~~347        &  **)&            \\
Acker et al. (1996) - Supplement 1 to SECGPN           &   ~~242        &  *)	&            \\
                                                       &   ~~142        &  **)&            \\
This catalogue - CGPN(2000)                            &   1510         &     &   245      \\
                                                       &	          &	&            \\
\hline						        	   	 
*) ~~true or probable                                     &	          &	&            \\
**) ~possible                                            &                &     &            \\
\end{tabular}

\vspace{.5cm}
 The distribution of the number of PNe discoveries is given in the following figure:
\protect\\


\subsection*{Individual distances}

Distance is one of the most fundamental parameters and appears in nearly all studies of PNe. 
There exist several \textbf{``statistical'' 
distances} based on various assumptions concerning the system of PNe. The differences of the statistical 
distances for a given object are large 
mainly because of the differences of individual objects called planetary nebulae. A better situation 
exists with \textbf{``individual'' distances} 
which are more or less reliable and are given basically by: trigonometric parallaxes of central stars, spectroscopic parallaxes of 
the companions of central stars, membership of clusters, expansion of nebulae and extinction of nebulae compared with the extinction in their vicinity. 
There are further methods which use 
some general conceptions, described mainly by Acker (1978), Sabbadin (1986) and Pottasch (1996), such 
as e.g. his ``gravity'' distances, depending on 
the correctness of the model atmosphere used. \\

We have divided the PNe with known individual distances into two categories according to the accuracy of the distances: the first category of distances is accurate up to 50\% or less; we estimate the accuracy of the second category of distances to be within 50-150\%. We would like to 
point out that the individual values given in the following table need not be wholly independent of each other, but we give them again in order to show the 
scatter of distances which is not small at all for some objects. However the present situation is much better than that some years ago because now almost 10\% of 
all planetaries have known individual distances (77 in the first and 63 in the second category).

\newpage
%\protect\footnotesize{
%\afterpage{\clearpage\input{distpn.tex}\clearpage}
\afterpage{\clearpage}
%}
\footnotesize{

\begin{longtable}{llllr@{}lr@{}l}
Design.      & Name         &\multicolumn{2}{c}{Cat.}&Dist. & &  Ref&    \\
             &              &            &           & [pc] & &     &    \\      
\hline                                         	     
             &              &            &           &      & &     &    \\
001$-$06.2~~~& SwSt 1~~~~~~~&            & 2         &~~~110&:& ~~~3&    \\	   
002+05.1     & NGC 6369     &            & 2         & 1800&: &    1&    \\
003$-$04.5   & NGC 6565     &  1         &           & 1100&  &    1&    \\
             &              &            &           & 2640&  &    4&    \\
             &              &            &           & 1000&  &    9&    \\
008+03.1     & NGC 6445     &  1         &           & 2200&  &    1&    \\
009$-$05.1   & NGC 6629     &            & 2         & 2400&  &    4&    \\
             &              &            &           & 1500&: &    1&    \\
010+00.1     & NGC 6537     &  1         &           & 2400&  &    1&    \\
010$-$01.1   & NGC 6578     &  1         &           & 2000&  &    1&    \\
011+05.1     & NGC 6439     &            & 2         & 1100&: &    1&    \\
011$-$00.2   & NGC 6567     &  1         &           & 1500&  &    1&    \\
             &              &            &           & 1530&  &    4&    \\
             &              &            &           & 1680&  &    9&    \\
             &              &            &           & 1100&  &   12&    \\
023$-$02.1   & M 1-59       &  1         &           & 1400&  &    1&    \\
             &              &            &           & 1450&  &   12&    \\
025$-$04.2   & IC 1295      &  1         &           & 1450&  &    8&    \\
025$-$17.1   & NGC 6818     &            & 2         & 2200&: &    1&    \\
029$-$05.1   & NGC 6751     &            & 2         & 1450&: &    1&    \\
030+06.1*    & Sh 2-68      &  1         &           &  560&  &    8&    \\
             &              &            &           &  950&  &    6&c   \\
             &              &            &           &  550&  &    6&d   \\      
033$-$02.1   & NGC 6741     &  1         &           & 1400&  &    4&    \\
             &              &            &           & 1700&: &    1&    \\
033$-$06.1   & NGC 6772     &  1         &           & 1400&  &    1&    \\  	
034+11.1     & NGC 6572     &  1         &           &  500&  &    1&    \\  	
             &              &            &           &  940&  &    5&    \\  	
034$-$06.1   & NGC 6778     &            & 2         & 1000&: &    1&    \\   
034$-$10.1*  & HtDe 11      &  1         &           & 1000&  &    8&    \\  
036$-$57.1   & NGC 7293     &  1         &           &  100&  &    1&    \\
             &              &            &           &  210&  &    2&    \\
             &              &            &           &  300&  &    4&    \\
             &              &            &           &  180&  &    6&b   \\
             &              &            &           &  320&  &    6&c   \\
037$-$06.1   & NGC 6790     &            & 2         & 1100&: &    1&    \\
037$-$34.1   & NGC 7009     &            & 2         & 2500&  &    4&    \\
             &              &            &           &  600&: &    5&    \\
             &              &            &           &  500&: &    1&    \\
041$-$02.1   & NGC 6781     &            & 2         & 1600&: &    1&    \\
043+37.1     & NGC 6210     &  1         &           & 1570&  &    5&    \\
045$-$04.1   & NGC 6804     &            & 2         & 1700&: &    1&    \\
046+03.1     & Sh 2-78      &  1         &           & 1420&  &    8&    \\
046$-$04.1   & NGC 6803     &            & 2         & 2500&: &    1&    \\
047+42.1     & A 39         &            & 2         & 1580&  &    6&c   \\
051+09.1     & Hu 2-1       &            & 2         &   78&: &    3&    \\
051+02.1     & IRAS1912+17  &  1         &           &  700&  &    6&a   \\
054$-$12.1   & NGC 6891     &            & 2         & 3800&  &    4&    \\
060$-$03.1   & NGC 6853     &  1         &           &  250&  &    1&    \\
             &              &            &           &  380&  &    2&    \\
             &              &            &           &  550&  &    8&    \\
             &              &            &           &  400&  &    6&c   \\
             &              &            &           &  310&  &    6&d   \\
060$-$07.1   & He 1-5       &  1         &           &  220&  &    3&    \\
060$-$07.2   & NGC 6886     &  1         &           & 1700&  &    1&    \\
063+13.1     & NGC 6720     &  1         &           &  400&  &    1&    \\
             &              &            &           &  700&  &    2&    \\
             &              &            &           &  990&  &    8&    \\
             &              &            &           &  700&  &    6&c   \\
             &              &            &           &  500&  &    6&d   \\
             &              &            &           &  400&: &   12&    \\
064+05.1     & BD+30 3639   &            & 2         &  600&  &    1&    \\
             &              &            &           & 2090&  &    5&    \\
065$-$27.1   & Ps 1         &  1         &           & 9600&  &    1&    \\
             &              &            &           &10100&  &    4&    \\
069$-$02.1   & NGC 6894     &  1         &           & 1500&  &    1&    \\
             &              &            &           & 1040&  &    4&    \\
072$-$17.1   & A 74         &  1         &           &  750&  &    2&    \\
             &              &            &           &  750&  &    8&    \\
             &              &            &           & 1000&  &    6&c   \\
             &              &            &           &  600&  &    6&d   \\
080$-$10.1*  & RXJ 2117+34  &  1         &           &  750&  &    6&c   \\
             &              &            &           &  200&  &    6&d   \\
082+07.1     & NGC 6884     &  1         &           & 1800&  &    1&    \\
084$-$03.1   & NGC 7027     &  1         &           & 1000&  &    1&    \\     
             &              &            &           &  840&  &    5&    \\
             &              &            &           &  790&  &    6&b   \\
085+52.1*    & PG 1520+525  &            & 2         &  600&  &    6&c   \\
086$-$08.1   & Hu 1-2       &            & 2         & 1600&: &    1&    \\
088$-$01.1   & NGC 7048     &            & 2         & 1900&: &    1&    \\
             &              &            &           & 2500&: &   12&    \\
089+00.1     & NGC 7026     &  1         &           & 2200&  &    1&    \\
             &              &            &           & 1450&  &    4&    \\
089$-$05.1   & IC 5117      &            & 2         & 2100&: &    1&    \\
093+05.2     & NGC 7008     &            & 2         & 1100&: &    1&    \\
094+27.1     & K 1-16       &            & 2         & 1700&  &    6&c   \\
             &              &            &           &  200&  &    6&d   \\
100$-$05.1   & IC 5217      &            & 2         & 1300&: &    1&    \\
104+07.1     & NGC 7139     &            & 2         & 1100&: &    1&   \\
104$-$29.1   & Jn 1         &            & 2         &  110&  &    6&c  \\
             &              &            &           &  500&  &    6&d  \\
106$-$17.1   & NGC 7662     &       1    &           &  650&  &    1&   \\
             &              &            &           &  960&  &    5&   \\
             &              &            &           &  790&  &    6&b  \\
107+07.1*    & IsWe 2       &       1    &           &  600&  &    6&c  \\
             &              &            &           &  700&  &    6&d  \\
             &              &            &           & 1490&  &    8&   \\
107+02.1     & NGC 7354     &            & 2         & 1500&  &    1&   \\
             &              &            &           & 3430&  &    4&   \\
111+11.1*    & DeHt 5       &       1    &           &  440&  &    8&   \\
             &              &            &           &  460&  &    6&c  \\
             &              &            &           &  250&  &    6&d  \\
118$-$74.1   & NGC 246      &       1    &           &  500&  &    1&   \\
             &              &            &           &  630&: &    3&   \\
             &              &            &           &  430&  &    4&   \\
             &              &            &           &  570&  &    5&   \\
             &              &            &           &  470&  &    6&a  \\
             &              &            &           &  590&  &    6&b  \\
             &              &            &           &  420&  &    6&c  \\
120+18.1*    & Sh 2-174     &       1    &           &  500&  &    8&   \\
             &              &            &           &  520&  &    6&c  \\
             &              &            &           &  350&  &    6&d  \\
120+09.1     & NGC 40       &       1    &           &  900&  &    1&   \\
120$-$05.1   & Sh 2-176     &       1    &           &  630&  &    6&c  \\
             &              &            &           &  480&  &    6&d  \\
124+10.1     & EL 0103+73   &       1    &           &  510&  &    6&c  \\
             &              &            &           &  600&  &    6&d  \\
             &              &            &           &  480&  &    8&   \\
125$-$47.1   & PHL 932      &            & 2         &  110&  &    3&   \\
             &              &            &           &  520&  &    6&c  \\
128$-$04.1*  & S 22         &       1    &           &  600&  &    8&     \\
             &              &            &           &  660&  &    6&c    \\
         &             &   &   &  600&  &    6&d  \\
         &             &   &   &  800&  &   11&   \\
130+01.1 & IC 1747     &  1&   & 2400&  &    1&   \\
         &             &   &   & 2200&  &    4&   \\
         &             &   &   & 2500&  &   12&   \\
130$-$10.1 & NGC 650-1 &   & 2 & 1550&  &    8&   \\
         &             &   &   &  780&  &    6&c  \\
         &             &   &   &  500&  &    6&d  \\
         &             &   &   & 1200&: &    1&   \\
138+02.1 & IC 289      &   & 2 & 1870&  &    4&   \\
144+06.1 & NGC 1501    &   & 2 & 1500&: &    1&   \\
148+57.1 & NGC 3587    &  1&   &  500&  &    1&   \\
         &             &   &   &  425&  &    6&b  \\
149$-$03.1*& IsWe 1    &  1&   &  550&: &    6&c  \\
         &             &   &   &  450&  &    6&d  \\
         &             &   &   &  480&  &    8&   \\
149$-$09.1*& HtDe 3    &  1&   & 1130&  &    8&   \\
         &             &   &   &  510&  &    6&c  \\
         &             &   &   &  500&  &    6&d  \\
         &             &   &   &  800&  &   11&   \\
156+12.1*& HtDe 4      &  1&   &  320&  &    6&c  \\
         &             &   &   &  500&  &    6&d  \\
156$-$13.1*& HtWe 5    &  1&   &  470&  &    6&c  \\
         &             &   &   &  450&  &    6&d  \\
158+17.1 & PuWe 1      &  1&   &  430&  &    2&   \\
         &             &   &   &  280&  &    6&c  \\
         &             &   &   &  500&  &    6&d  \\
         &             &   &   &  470&  &    8&   \\
158+00.1*& Sh 2-216    &  1&   &  130&  &    2&   \\
         &             &   &   &  130&  &    6&c  \\
         &             &   &   &  120&  &    6&d  \\
         &             &   &   &  110&  &    8&   \\
164+31.1 & JnEr 1      &   & 2 &  900&  &    6&c  \\
165$-$15.1 & NGC 1514  &  1&   &  600&  &    1&   \\
         &             &   &   &  185&  &    3&   \\
         &             &   &   &  660&  &    4&   \\
         &             &   &   &  400&  &    6&a  \\
         &             &   &   &  740&  &    6&d  \\
166+10.1 & IC 2149     &   & 2 &  900&: &    1&   \\
194+02.1 & J 900       &   & 2 & 1600&: &    1&   \\
196$-$10.1 & NGC 2022  &   & 2 & 1500&: &    1&   \\
197+17.1 & NGC 2392    &  1&   &  600&  &    1&   \\
         &             &   &   & 2700&  &    4&   \\
         &             &   &   & 1600&  &    5&   \\
197$-$06.1 & WeDe 1    &  1&   &  800&  &    6&c  \\
         &             &   &   &  600&  &    6&d  \\
         &             &   &   &  880&  &    8&   \\
204+04.1 & K 2-2       &  1&   & 1170&  &    8&   \\
205+14.1*& A 21        &  1&   &  540&  &    2&	\\
         &             &   &   &  630&  &    6&c	\\
         &             &   &   &  500&  &    6&d	\\
         &             &   &   &  530&  &    7&	\\
         &             &   &   &  600&  &    8&	\\
206$-$40.1 & NGC 1535  &   & 2 & 2700&  &    4&	\\
         &             &   &   & 1500&: &    1&	\\
211+22.1*& BN 0808+11  &  1&   &  600&  &    6&c	\\
         &             &   &   &  970&  &    6&d	\\
215+03.1 & NGC 2346    &  1&   &  800&  &    1&	\\
         &             &   &   &  690&  &    6&a	\\
         &             &   &   &  900&  &    6&d	\\
         &             &   &   & 1060&  &    9&	\\
215$-$24.1 & IC 418    &   & 2 & 2000&  &    4&	\\
         &             &   &   &  330&: &    1&	\\
215$-$30.1 & A 7       &   & 2 &  550&  &    6&c	\\
217+14.1 & A 24        &   & 2 &  320&: &    2&	\\
         &             &   &   &  800&  &    6&d	\\
219+31.1 & A 31        &   & 2 &  210&: &    2&	\\
220$-$53.1 & NGC 1360  &  1&   &  350&  &    3&	\\
         &             &   &   &  670&  &    4&	\\
         &             &   &   &  420&  &    6&c	\\
         &             &   &   &  550&: &    1&	\\
221+46.1*& BN 0950+13  &   & 2 &  525&  &    6&c	\\
221$-$12.1 & IC 2165   &   & 2 & 2400&: &    1&	\\
224+01.1 & We 1-6      &   & 2 & 1100&  &   11&	\\
231+04.2 & NGC 2438    &   & 2 & 2000&: &    1&	\\
234+02.1 & NGC 2440    &  1&   & 2100&  &    1&	\\
         &             &   &   &  885&  &    4&	\\
         &             &   &   & 1000&: &   12&	\\
         &             &   &   & 2190&  &    9&	\\
238+34.1 & A 33        &   & 2 & 1260&  &    6&c	\\
243$-$01.1 & NGC 2452  &  1&   & 3500&  &    1&	\\
         &             &   &   & 3230&  &    4&	\\
         &             &   &   & 3570&  &    9&	\\
         &             &   &   & 3100&: &   12&	\\
244+12.1 & A 29        &  1&   &  460&: &    2&	\\
         &             &   &   &  300&  &    7&	\\
259+00.1 & He 2-11     &   & 2 &  770&: &    7&	\\
261+32.1 & NGC 3242    &  1&   & 2000&  &    4&	\\
         &             &   &   &  465&  &    5&	\\
         &             &   &   &  420&  &    6&b	\\
         &             &   &   &  500&: &    1&	\\
261+08.1 & NGC 2818    &  1&   & 2300&  &    4&	\\
265+04.1 & NGC 2792    &  1&   & 2000&  &    1&   \\
         &             &   &   & 1930&  &    4&   \\
         &             &   &   & 1910&  &    9&   \\
272+12.1 & NGC 3132    &  1&   &  700&  &    1&   \\
         &             &   &   &  510&  &    6&a  \\
         &             &   &   &  420&  &    6&d  \\
         &             &   &   &  540&  &    9&   \\
273+06.1*& HBDS 1      &  1&   & 1000&  &    6&c  \\
         &             &   &   &  900&  &    6&d  \\
278$-$05.1 & NGC 2867  &   & 2 & 2000&: &    1&   \\
279$-$03.1 & He 2-36   &  1&   &  780&  &    6&a  \\
281$-$05.1 & IC 2501   &   & 2 & 1700&: &    1&   \\
283+09.1*& ESO-215-04  &  1&   &  865&  &    6&c  \\
         &             &   &   & 1050&  &    6&d  \\
285$-$14.1 & IC 2448   &   & 2 & 4500&  &    4&   \\
         &             &   &   &  830&  &   10&   \\
286$-$04.1 & NGC 3211  &  1&   & 2000&  &    1&   \\
         &             &   &   & 1530&  &    4&   \\
         &             &   &   & 1910&  &    9&   \\
286$-$29.1 & K 1-27    &   & 2 & 1200&  &    6&c  \\
294+43.1 & NGC 4361    &  1&   & 1300&  &    4&   \\
         &             &   &   & 1200&  &    6&c  \\
         &             &   &   & 1400&: &    1&   \\
294+04.1 & NGC 3918    &  1&   & 2000&  &    1&   \\
         &             &   &   & 1790&  &    4&   \\
         &             &   &   & 2240&  &    9&   \\
298$-$04.1 & NGC 4071  &   & 2 & 1500&: &    1&   \\
303+40.1 & A 35        &  1&   &  130&  &    3&   \\
         &             &   &   &  200&  &    6&a  \\
         &             &   &   &  360&: &    1&   \\
307$-$03.1 & NGC 5189  &  1&   & 1600&  &    1&   \\
         &             &   &   & 1730&  &    9&   \\
307$-$04.1 & MyCn 18   &   & 2 & 1400&: &    1&   \\
309$-$04.2 & NGC 5315  &  1&   & 2500&  &    1&   \\
         &             &   &   & 2820&  &    4&   \\
         &             &   &   & 2620&  &    9&   \\
315$-$13.1 & He 2-131  &  1&   &  610&  &    4&   \\
         &             &   &   &  590&  &    9&   \\
         &             &   &   &  710&  &   10&   \\
316+08.1 & He 2-108    &   & 2 & 8300&  &    4&   \\
318+41.1 & A 36        &  1&   &  240&  &    3&   \\
         &             &   &   &  600&  &    6&c  \\
         &             &   &   &  400&: &    1&   \\
320$-$09.1 & He 2-138  &   & 2 & 5000&  &    4&   \\
322$-$02.1 & Mz 1      &   & 2 & 1100&: &    1&   \\
327+10.1 & NGC 5882    &  1&   & 1100&: &    1&   \\
         &             &   &   &  560&  &   10&   \\
329+02.1 & Sp 1        &   & 2 & 1500&: &    1&   \\
331$-$01.1 & Mz 3      &   & 2 & 4500&  &   12&   \\
334$-$07.1*&CPD-53 8315&  1&   &  130&  &    3&   \\
335+12.1*& DS 2        &  1&   &  850&  &    6&c  \\
         &             &   &   &  720&  &    6&d  \\
338$-$08.1 & NGC 6326  &   & 2 &  800&: &    1&   \\
339+88.1 & LoTr 5      &  1&   &  420&  &    6&a  \\
341+05.1 & NGC 6153    &   & 2 & 1700&: &    1&   \\
342+10.1 & NGC 6072    &   & 2 & 1800&: &    1&   \\
345+00.1 & IC 4637     &   & 2 & 1600&  &    4&   \\
345$-$08.1 & Tc 1      &   & 2 & 3800&  &    4&   \\
         &             &   &   &  590&  &   10&   \\
349+01.1 & NGC 6302    &  1&   & 1600&  &    5&   \\
350+04.1 & H 2-1       &   & 2 & 4600&  &    4&   \\
358$-$00.2 & M 1-26    &  1&   & 1900&  &    4&   \\
         &             &   &   & 1800&: &    1&   \\
         &             &   &   & 1800&  &   12&   \\
358$-$07.1 & NGC 6563  &   & 2 &  700&: &    1&   \\
359$-$00.1 & Hb 5      &  1&   & 2000&: &    1&   \\
         &             &   &   & 2000&  &   12&   \\
         &             &   &   & 1500&: &    1&   \\
\end{longtable}			        	   
%\newpage
\footnotesize{
REFERENCES:
\begin{enumerate}
\item Sabbadin F., 1986, A\&AS 64, 579. (Table 1: column 10) 
\item Harris H.C., Dahn C.C., Monet D.G., Pier J.R., 1997, Proc. IAU Symp.180, 
       40.\\ (Table 2: Trigonometric parallaxes)
\item Acker A., Fresneau A., Pottasch S.R., Jasniewicz G., 1998, A\&A 337, 253.\\
       (Table 1: HIPPARCOS, reliable trigonometric parallaxes)  
\item Cahn J.H., Kaler J.B., Stanghellini L., 1992, A\&AS 94, 399.  
       (Table 3: Calibration nebulae)
\item Terzian Y., 1997, Proc. IAU Symp.180, 29. (Table 1: Expansion distances 
        - mean)
\item Pottasch S.R., 1996, A\&A 307, 561: 
\begin{enumerate}
\item (Table 3: Spectroscopic distances)
\item (Table 5: Expansion distances)
\item (Table 6: ``Gravity'' distances)
\item (Table 8: Extinction distances)
\end{enumerate}
\item Gutierrez-Moreno A., Anguita C., Loyola P., Moreno H., 1999, PASP 111, 
       1163. (Table 3: Trigonometric parallaxes)
\item Napiwotzki R., Sch\"onberner D., 1995, A\&A 301, 545. (Table 6: Na D 
       extinction distances)
\item Gathier R., Pottasch S.R., Pel J.W., 1986, A\&A 157, 171. (Table 12:
       Extinction distances)
\item Martin W., 1994, A\&A 281, 526. (Table 4: Extinction distances)
\item Saurer W., 1995, A\&A 297, 261. (Extinction distances)
\item Acker A., 1978, A\&AS 33, 367. (Table 4: Extinction distances D(ABS))
\end{enumerate}
}
\normalsize
\protect\\
\subsection*{Binary central stars}


Altogether 35 central stars have been recognized as binaries up till now,
partly resolved as visual binaries (see Ciardullo et al., 1999) and partly known as close binary nuclei (Bond, 2000). They are listed in the following
table; observers should be encouraged to make more observations of
central stars because the percentage of known binaries is much smaller compared with that of the ``common'' stars. There would also be an opportunity to get further 
individual distances of PNe.  \\


\footnotesize{

\begin{longtable}{lll}
Design.  &Name          & Remarks                                                       \\
\hline
         &           & ~~~~~~~~~~~~~~~~~~~~~~~~~~~~~~~~~~~~~~~~~~~~~~~~~~~~~~~~~~~~~~~~ \\
005$-$08.1  & Hf 2-2      &  Close binary nucleus, orbital period 0.399 days.           \\
009$+$10.1  & A 41        &  Close binary nucleus, orbital period 0.113 days. 	        \\ 
017$-$21.1  & A 65        &  Close binary nucleus, orbital period 1.00 days. 		\\ 
045$+$24.1  & K 1-14      &  On the POSS a pair of stars separated by 9.1 arcsec	\\ 
          &             &    is in the centre of the nebula. The actual nucleus		\\ 
          &             &    is a star lying 2.4 arcsec southwest of the brighter 	\\ 
          &             &    POSS star; it is a probable physical pair                  \\
          &             &    (component d=0.36 arcsec, p.a.=242 deg). 	        \\
053$-$03.1  & A 63        &  Close binary nucleus, eclipsing binary, orbital period	\\
          &             &    0.465 days. The CS is a possible physical pair		\\
          &             &    (component d=2.82 arcsec, p.a.=94 deg).	        \\
055$+$16.1  & A 46        &  Close binary nucleus, eclipsing binary, orbital period     \\
          &             &    0.472 days.                                                \\
093$+$05.2  & NGC 7008    &  The CS is a probable physical pair (component		\\
          &             &    d=0.42 arcsec, p.a.=241 deg).				\\
118$-$74.1  & NGC 246     &  The CS is a physical pair (component d=3.81 arcsec,	\\
          &             &     p.a.=130.3 deg). 				                \\
130$-$10.1  & NGC 650-1   &  The CS is a doubtful physical pair. The companion          \\
          &             &    itself is almost certainly a physical system		\\
          &             &    (d=0.16 arcsec).					\\
136$+$05.1  & HEFE 1      &  Close binary nucleus, orbital period 0.582 days.		\\
144$+$65.1  & BE UMa      &  Close binary nucleus, eclipsing binary, orbital period     \\
          &             &    2.29 days.                                                 \\
158$+$17.1  & PuWe 1      &  The CS is a doubtful physical pair. The component	        \\
          &             &    itself is a close pair (d=0.6).                       \\
197$+$17.1  & NGC 2392    &  The CS is a possible physical pair (component		\\
          &             &    d=2.65 arcsec, p.a.=213 deg).				\\
206$-$40.1  & NGC 1535    &  The CS is a probable physical pair (component		\\
          &             &    d=1.04 arcsec, p.a.=334 deg).				\\
208$+$33.1  & A 30        &  The CS is a possible physical pair (component		\\
          &             &    d=5.25 arcsec, p.a.=144 deg).				\\
215$+$03.1  & NGC 2346    &  Close binary nucleus, orbital period 15.99 days.  	        \\
215$-$30.1  & A 7         &  The CS is a possible physical pair (component		\\
          &             &    d=0.91 arcsec, p.a.=250 deg).				\\
217$+$14.1  & A 24        &  The CS is prob. an optical double (component		\\
          &             &    d=3.33 arcsec, p.a.=159 deg).  			\\
219$+$31.1  & A 31        &  The CS is a probable physical pair (component		\\
          &             &    d=0.26 arcsec, p.a.=242 deg).				\\
228$-$22.1  & DeHt 1      &  LoTr 1. Close binary nucleus, orbital period unknown.      \\ 
238$+$34.1  & A 33        &  The CS is probably a physical pair (component		\\
          &             &    d=1.82 arcsec, p.a.=209 deg). 			\\
239$+$13.1  & NGC 2610    &  The CS is a possible physical pair (component		\\
          &             &    d=0.61 arcsec, p.a.=277 deg).				\\
253$+$10.1  & K 1-2       &  Close binary nucleus, orbital period 0.676 days.		\\
272$+$12.1  & NGC 3132    &  The CS is probably a physical pair (component		\\
          &             &    d=1.71 arcsec, p.a.=47 deg).				\\
283$+$25.1  & K 1-22      &  The CS is a probable physical pair (component		\\
          &             &    d=0.35 arcsec, p.a.=218 deg).				\\
283$+$09.1* & ESO-215-04  &  Close binary nucleus, orbital period 0.357 days.		\\
286$-$29.1  & K 1-27      &  The CS is a probable physical pair (component		\\
          &             &    d=0.56 arcsec, p.a.=315 deg).				\\
303$+$40.1  & A 35        &  Close binary nucleus, orbital period unknown. 		\\
311$+$02.2  & SuWt 2      &  Close binary nucleus, orbital period 2.45 days.		\\
329$+$02.1  & Sp 1        &  Close binary nucleus, orbital period 2.91 days. 		\\
329$-$02.2  & Mz 2        &  The CS is a probable physical pair (component		\\
          &             &    d=0.28 arcsec, p.a.=218 deg).				\\
335$-$03.1  & HtTr 4      &  Close binary nucleus, orbital period 1.74 days.            \\
339$+$88.1  & LoTr 5      &  Close binary nucleus, orbital period unknown.		\\
342$-$14.1  & Sp 3        &  The CS is a probable physical pair (component		\\
          &             &    d=0.31 arcsec, p.a.=42 deg).				\\
345$+$00.1  & IC 4637     &  The CS is a possible physical pair (component		\\
          &             &    d=2.42 arcsec, p.a.=330 deg).				\\
	  &	      &								        \\
\end{longtable}
}
\normalsize
\protect\vspace{-0.3cm}
\subsection*{Variable objects}

There are altogether 123 PNe (besides 26 variables belonging to the objects, which were removed from the list of PNe) listed in the GCVS + supplementary lists (Kholopov et al., 1985-1990) or in the catalogue of suspected variable stars (Kholopov et al., 1982). The fact that 54\% of the above objects were classified as suspected variables reflects the difficulties of recognizing PNe as variable objects (mainly due to the disturbing effect of the nebulae). Moreover 4 objects of the 56 known variables appeared to be constant. The remaining variables are in most cases symbiotic variables of Z And type (n=10) and unique objects (n=8). Only 3 objects (A 46, A 63, BE UMa) are Algol-type eclipsing binaries. In the following summary all variable PNe (also misclassified objects) are listed including their type of variability according to GCVS. The main purpose of this summary is to encourage observers to monitor the PNe in order to find further variables, which would be very desirable.
               

\footnotesize{
\begin{longtable}{llll}
~~~~~~~~~~~~~&~~~~~~~~~~~~~~~&~~~~~~~~~~~~~~~~~&~~~~~~~~~~~~~~~~\\
\multicolumn{4}{l}{{\bf A) Planetary nebulae:}}                     \\
             &              &                  &              \\
Design.      & Name         & Variable         &Type 	      \\
\hline	    	        		         		      
             &              &                  &              \\
002$+$05.1   & NGC 6369     & V 2310 Oph       &ZZ:	      \\
002$-$05.1*  & He 2-370     & V 2756 Sgr:      &ZAND	      \\
005$-$05.2*  & He 2-390     & V 3929 Sgr       &ZAND: 	      \\
007$+$01.1   & Hb 6         & AS Sgr           &~~~CST:	      \\
007$+$01.2*  & M 3-18       & V 2416 Sgr       &ZAND	      \\
008$+$03.2*  & Th 4-4       & V 4141 Sgr       &ZAND:         \\
009$+$10.1   & A 41         & MT Ser           &ELL/PN	      \\
010$+$04.2   & V 4334 Sgr   & V 4334 Sgr       &?             \\
011$-$07.1*  & V 348 Sgr    & V 348 Sgr        &unique	      \\
037$-$03.3*  & K 3-25       & V 352 Aql        &NL	      \\
037$-$05.1   & A 58         & V 605 Aql        &unique        \\
048$+$04.1*  & K 4-12       & V 1413 Aql       &ZAND+E        \\
053$-$03.1   & A 63         & UU Sge           &EA/D/PL       \\
055$+$16.1   & A 46         & V 477 Lyr        &EA/PN         \\
058$-$10.1   & IC 4997      & QV Sge           &unique        \\
060$-$07.1   & He 1-5       & FG Sge           &unique        \\
061$-$09.1   & NGC 6905     & NT Del           &ZZ:           \\
063$-$12.1*  & He 2-467     & LT Del           &ZAND          \\
064$+$05.1   & BD$+$30 3639   & V 1966 Cyg       &E:/PN       \\
075$+$05.1*  & V 1016 Cyg   & V 1016 Cyg       &NC+M   	      \\
076$-$05.1*  & LSII$+$34 26   & V 1853 Cyg       &ACYG        \\
080$-$10.1*  & RXJ 2117$+$34  & V 2027 Cyg       &ZZO	      \\
094$+$27.1   & K 1-16       & DS Dra           &ZZO	      \\
120$+$09.1   & NGC 40       & V 400 Cep        &WR	      \\
133$-$08.1*  & M 1-2        & V 471 Per        &~~~CST	      \\
136$+$05.1   & HEFE 1       & V 664 Cas        &R	      \\
144$+$65.1   & BE UMa       & BE UMa           &EA/WD/D       \\
144$+$06.1   & NGC 1501     & CH Cam           &ZZ	      \\
166$-$06.1*  & CRL 618      & V 353 Aur        &E	      \\
174$-$14.1   & H 3-29       & GL Tau           &~~~CST:	      \\
215$+$03.1   & NGC 2346     & V 651 Mon        &unique	      \\
215$-$24.1   & IC 418       & ZZ Lep           &NL:	      \\
228$-$22.1   & LoTr 1       & AB Lep           &RS:	      \\
243$-$01.1   & NGC 2452     & V 354 Pup        &ZZ	      \\
253$+$10.1   & K 1-2        & VW Pyx           &unique	      \\
274$+$02.1*  & He 2-34      & KM Vel           &M	      \\
280$-$02.1*  & He 2-38      & V 366 Car        &M+ZAND        \\
283$+$09.1*  & ESO-215-04   & KV Vel (not KU)  &R/PN	      \\
299$+$00.1*  & BI Cru       & BI Cru           &ZAND	      \\
303$+$40.1   & A 35         & LW Hya           &R:	      \\
307$-$03.1   & NGC 5189     & KN Mus           &ZZO	      \\
311$+$03.1*  & He 2-101     & V 704 Cen        &L:	      \\
312$-$00.1*  & V 417 Cen    & V 417 Cen        &L	      \\
312$-$02.1*  & He 2-106     & V 835 Cen        &M	      \\
315$+$09.1*  & He 2-104     & V 852 Cen        &M:	      \\
320$-$09.1   & He 2-138     & MM TrA           &BE	      \\
326$-$10.1*  & Cn 1-2       & KX TrA           &ZAND:         \\
327$-$04.1*  & He 2-147     & V 347 Nor        &M:	      \\
331$-$12.1   & CPD-59 6926  & V 839 Ara        &BE	      \\
332$-$09.1*  & CPD-56 8032  & V 837 Ara        &unique	      \\
339$+$88.1   & LoTr 5       & IN Com           &R:/PL	      \\
344$-$08.1*  & PC 18        & AE Ara           &unique	      \\
351$+$03.1*  & H 2-2        & V 455 Sco        &ZAND	      \\
355$-$06.1   & M 3-21       & V 567 Sgr        &~~~CST	      \\
356$-$07.1*  & He 2-349     & V 615 Sgr        &ZAND	      \\
358$-$21.1   & IC 1297      & RU CrA           &L:	      \\
	     &	            &		       &	      \\
009$-$05.1   & NGC 6629     & NSV 10766        &	      \\
010$+$18.1*  & K 2-8        & NSV 20901        &	      \\
010$-$06.1   & IC 4732      & NSV 11021        &	      \\
012$-$07.1*  & He 2-417     & NSV 24553        &        \\
013$+$05.1*  & Sa 3-96      & NSV 24041        &	\\
014$-$05.1   & V-V 3-5      & NSV 24521        &	\\
017$-$10.1   & A 51         & NSV 11642        &	\\
023$-$01.1*  & K 3-9        & NSV 24539        &	\\
025$+$40.1   & IC 4593      & NSV 7526	       &	\\
025$-$17.1   & NGC 6818     & NSV 24857        &	\\
034$+$11.1   & NGC 6572     & NSV 24329        &	\\
036$+$17.1   & A 43         & NSV 24029        &	\\
036$-$01.1   & Sh 2-71      & NSV 24661        &	\\
037$-$06.1   & NGC 6790     & NSV 11959        &	\\
037$-$34.1   & NGC 7009     & NSV 25448        &	\\
045$+$24.1   & K 1-14       & NSV 23577        &	\\
045$-$04.1   & NGC 6804     & NSV 24799        &	\\
046$-$04.1   & NGC 6803     & NSV 24796        &	\\
047$+$42.1   & A 39         & NSV 7743	        &	\\
051$+$09.1   & Hu 2-1       & NSV 24612        &	\\
054$-$12.1   & NGC 6891     & NSV 25070        &	\\
061$+$08.1   & K 3-27       & NSV 24721        &	\\
061$+$02.1   & He 2-442     & NSV 12270        &	\\
063$+$13.1   & NGC 6720     & NSV 11500 (cst?) &	\\
066$-$28.1   & NGC 7094     & NSV 25698        &	\\
072$-$17.1   & A 74         & NSV 13633        &	\\
075$-$04.1*  & He 2-468     & NSV 25253        &	\\
077$+$14.1   & A 61         & NSV 11917        &	\\
081$-$14.1   & A 78         & NSV 13798        &	\\
082$+$07.1   & NGC 6884     & NSV 25033        &	\\
083$+$12.1   & NGC 6826     & NSV 12382        &	\\
089$+$00.1   & NGC 7026     & NSV 25472        &	\\
089$-$02.1*  & M 1-77       & NSV 25551        &	\\
106$-$17.1   & NGC 7662     & NSV 14555        &	\\
111$-$02.1   & Hb 12        & NSV 26083        &	\\
118$-$74.1   & NGC 246      & NSV 15170        &	\\
123$+$34.1   & IC 3568      & NSV 19422        &	\\
138$+$02.1   & IC 289       & NSV 1056	        &	\\
146$+$07.1   & M 4-18       & NSV 15974        &	\\
164$+$31.1   & JnEr 1       & NSV 17664        &	\\
165$-$15.1   & NGC 1514     & NSV 15907        &	\\
166$+$10.1   & IC 2149      & NSV 16733        &	\\
189$+$19.1   & NGC 2371-2   & NSV 17461        &	\\
193$-$09.1   & H 3-75       & NSV 16624        &	\\
197$+$17.1   & NGC 2392     & NSV 3604	        &	\\
197$-$03.1   & A 14         & NSV 2861	        &	\\
208$+$33.1   & A 30         & NSV 4241	        &	\\
214$+$07.1   & A 20         & NSV 3563	        &	\\
232$-$04.1   & M 1-11       & NSV 3447	        &	\\
261$+$32.1   & NGC 3242     & NSV 4848	        &	\\
278$-$05.1   & NGC 2867     & NSV 18182        &	\\
279$-$03.1   & He 2-36      & NSV 18259        &	\\
286$+$02.1   & He 2-55      & NSV 18536        &	\\
294$+$43.1   & NGC 4361     & NSV 19380        &	\\
311$+$02.2   & SuWt 2       & NSV 19992        &	\\
315$-$13.1   & He 2-131     & NSV 20337        &	\\
318$+$41.1   & A 36         & NSV 6376	        &	\\
319$+$06.1   & He 2-112     & NSV 20150        &	\\
322$-$05.1   & NGC 5979     & NSV 20395        &	\\
325$+$04.2*  & He 2-127     & NSV 20307        &	\\
330$+$04.1*  & Cn 1-1       & NSV 20412        &	\\
339$+$00.1*  & He 2-176     & NSV 20734        &	\\
346$+$08.1*  & He 2-171     & NSV 20689        &	\\
353$-$04.1*  & H 1-36       & NSV 23921        &	\\
354$+$04.2*  & H 2-5        & NSV 21208        &	\\
358$+$01.3*  & H 1-25       & NSV 22772        &	\\
359$-$07.1   & M 2-32       & NSV 24340        &	\\
	     &	            &		       &	\\
\multicolumn{4}{l}{{\bf B) Not planetary nebulae (misclassified):}} 	\\
	     &	            &		       &		\\
Design.      & Name         & Variable         &		\\
\hline	    	        		         		  
	     &	            &		       &		\\
000$-$01.4M  & Bl 3-14      & NSV 24017        &		\\
001$+$04.1M  & MaC 1-5      & V 2116 Oph       &		\\
001$-$00.1M  & Bl 3-11      & V 4375 Sgr       &		\\
003$-$04.6M  & Ap 1-11      & V 2506 Sgr:      &  	    \\
004$+$06.1M  & H 2-12       & V 843 Oph        &		    \\
010$-$03.1M  & He 2-396     & V 3811 Sgr:      &		    \\
037$-$06.2M  & K 4-26       & V 850 Aql        &		    \\
047$-$03.1M  & K 4-27       & V 976 Aql:       &		    \\
050$+$03.1M  & M 1-67       & QR Sge (NSV 11797) &	    \\
050$-$03.1M  & K 4-31       & V 998 Aql        &		    \\
055$-$01.2M  & K 3-44       & FT Sge	        &		    \\
064$+$00.1M  & K 3-47       & AA Vul	        &		    \\
104$-$06.1M  & M 2-54       & NSV 25967        &		    \\
108$-$05.1M  & K 4-46       & LL Cas	        &		    \\
110$-$00.1M  & We 1-12      & V 807 Cas        &		    \\
195$-$00.1M  & Sh 2-266     & V 1308 Ori       &		    \\
235$+$01.1M  & V-V 1-7      & NSV 17554        &		    \\
241$-$07.1M  & M 4-1        & NSV 17408        &		    \\
266$+$02.1M  & Pe 2-3       & NSV 18145        &		    \\
289$-$00.1M  & AG CAR       & AG CAR	        &		    \\
299$-$00.1M  & He 2-80      & NSV 19372        &		    \\
302$-$00.1M  & He 2-87      & NSV 19475        &		    \\
326$-$01.1M  & He 2-139     & NSV 20427        &		    \\
338$+$05.2M  & He 2-156     & QS Nor (KZP 2635): &	    \\
354$+$04.2M  & H 2-5        & NSV 21208        &		    \\
358$-$01.2M  & Sa 3-80      & NSV 23955        &		    \\
\end{longtable}						   
						   
}						   

\normalsize
\subsection*{Planetary nebulae in clusters}

A summary of PNe in clusters is given as a compilation from the
existing literature. We would like to emphasize that only \textbf{positional coincidences} (PNe are in or near the clusters) are indicated. Due to the relatively short lifetime of planetary nebulae
the probability of a real coincidence PN-cluster is rare, and
the question of such a relation remains unanswered. This summary
should not replace thorough studies of distances and radial velocities made in the original literature, and it has an informative character only.\\

\footnotesize{
\begin{tabular}{lllll}
Design.   &Name                & Cluster         &\multicolumn{2}{c} {Cluster  design.}    \\ 
\hline
          &                    &                 &           &            \\  
002$+$01.4  & JaFu 1             & Pal 6           &  GCl-75   & C 1740$-$262 \\
002$-$02.2  & M 3-20             & An(Tr 31)       &  OCl-9    & C 1756$-$281 \\
009$-$07.1  & IRAS18333-2357     & NGC 6656 = M 22 &  GCl-99   & C 1833$-$239 \\
065$-$27.1  & Ps 1 (Kustner 648) & NGC 7078 = M 15 &  GCl-120  & C 2127$+$119 \\
172$+$00.1* & A 9                & NGC 1912 = M 38 &  OCl-433  & C 0525$+$358 \\
231$+$04.2  & NGC 2438           & NGC 2437 = M 46 &  OCl-601  & C 0739$-$147 \\
243$-$01.1  & NGC 2452           & NGC 2453        &  OCl-670  & C 0745$-$251 \\
261$+$08.1  & NGC 2818           & NGC 2818        &  OCl-743  & C 0914$-$364 \\
324$-$01.1  & He 2-133           & An(Lynga 5)     &  OCl-942  & C 1538$-$564 \\
327$-$05.1  & KsRm 1             & NGC 6087        &  OCl-948  & C 1614$-$577 \\
329$-$02.3  & HeFa 1             & NGC 6067        &  OCl-953  & C 1609$-$540 \\
347$+$01.1  & Vd 1-8 (Sa 2-167)  & NGC 6281        &  OCl-1003 & C 1701$-$378 \\
353$-$05.2  & JaFu 2             & NGC 6441        &  GCl-78   & C 1746$-$370 \\
359$-$01.3  & M 3-45             & An(Basel 5)     &  OCl-1038 & C 1749$-$300 \\
\end{tabular}
}
\normalsize
\subsection*{X-ray sources}
   
   Using the data of recent x-ray satellites (EXOSAT, Einstein, ASCA, ROSAT) 16 PNe
were identified as X-ray sources. X-ray emissions can in principle appear\\
1) as point sources originating in the photospheres or possibly in the coronal regions 
of hot central stars (T $\geqq$ 100000 K), and\\
2) as extended (diffuse) sources showing the existence of very hot (T $\geqq$ 10$^{6}$ K) nebular plasma and fast stellar winds.\\
Both emisssions together are sometimes called ``two-component model''.\\
   
   The evidence for extended emission is completely unclear and some detections reported 
earlier may be spurious (see Chu et al., 1993; Chu, Ho, 1955). The aim of the following
summary is to motivate further research of the X-ray emissions of PNe which can be 
very useful for our understanding of the origin of PNe.\\

\newpage
\footnotesize{

\begin{longtable}{lllr}
Design.  & Name          & Remarks                             & Ref  \\  
\hline	  		 				       
	 &               &                                     &      \\
036$-$57.1 & NGC 7293      & ROSAT PSPC point source             &   1  \\
         &               & Einstein source                     &   3  \\
         &               & ROSAT PSPC two components source    &   5  \\
         &               & ROSAT PSPC two components source    &   8  \\
         &               & EXOSAT LE source                    &  14  \\         
060$-$03.1 & NGC 6853      & ROSAT PSPC point source             &   1  \\
         &               & ROSAT PSPC extended source          &   2  \\
         &               & Einstein source                     &   3  \\
         &               & ROSAT PSPC:no extended source       &  10  \\
         &               & ROSAT:no extended source            &   7  \\
         &               & ROSAT source                        &  11  \\
         &               & EXOSAT LE source                    &  14  \\
064+05.1 & BD+30 3639    & ROSAT PSPC point source             &   1  \\
         &               & ROSAT PSPC extended source          &   2  \\
         &               & ROSAT extended source               &   6  \\
         &               & ROSAT HRI extended source           &  13  \\
080$-$10.1 & RXJ 2117+34   & ROSAT PSPC source                   &  16  \\
094+27.1 & K 1-16        & ROSAT PSPC source                   &  11  \\
096+29.1 & NGC 6543      & ROSAT PSPC extended source          &   1  \\
         &               & ROSAT PSPC extended source          &   2  \\
118$-$74.1 & NGC 246       & ROSAT PSPC point source             &   1  \\
         &               & Einstein point source               &   3  \\
         &               & ROSAT source                        &  11  \\
         &               & EXOSAT LE source                    &  14  \\
148+57.1 & NGC 3587      & ROSAT source                        &   1  \\
         &               & ROSAT PSPC two components source    &   8  \\
         &               & ROSAT PSPC point source             &   9  \\
         &               & EXOSAT LE source                    &  14  \\
198$-$06.1 & A 12          & ROSAT PSPC extended source          &   2  \\
         &               & ROSAT:no extended source            &   7  \\
206$-$40.1 & NGC 1535      & ROSAT PSPC two components source    &   8  \\
         &               & ROSAT PSPC:previous reports spurious&   9  \\
         &               & EXOSAT LE source                   &  14  \\
208+33.1 & A 30          & ROSAT PSPC extended source         &   1  \\
         &               & ROSAT PSPC extended source         &   7  \\
         &               & ROSAT HRI source                   &  12  \\
220$-$53.1 & NGC 1360      & ROSAT PSPC point source            &   1  \\
         &               & EXOSAT source                      &   4  \\
         &               & ROSAT source                       &  11  \\
         &               & EXOSAT LE source                   &  14  \\
238+34.1 & A 33          & Einstein source                    &   3  \\
294+43.1 & NGC 4361      & ROSAT PSPC extended source         &   2  \\
         &               & ROSAT:no extended source           &   7  \\
         &               & EXOSAT LE source                   &  14  \\
318+41.1 & A 36          & ROSAT source                       &   1  \\
         &               & ROSAT PSPC two components source   &   8  \\
         &               & EXOSAT LE source                   &  14  \\             
339+88.1 & LoTr 5        & ROSAT PSPC point source            &   1  \\
         &               & ROSAT PSPC extended source         &   2  \\
         &               & ROSAT:no extended source           &   7  \\
         &               & ROSAT source                       &  11  \\
         &               & EXOSAT source                      &  15  \\
\end{longtable}	

\newpage
REFERENCES:
\begin{enumerate}
\item   Conway G.M., Chu Y.-H., 1997, IAU Symp. No.180, 214 (poster).
\item   Kreysing H.C., Diesch C., Zwiegle J. et al., 1992, A\&A 264, 623
      (IAU Symp. No.155, 197; 1993).
\item   Tarafdar S.P., Apparao K.M.V., 1988, ApJ 327, 342.
\item   de Korte P.A.J., Claas J.J., Jansen F.A., McKechnie S.P., 1984, in
      X-Ray Astronomy '84 (eds. M. Oda, R. Giacconi), p.35. (see also Ref.3)
\item   Leahy D.A., Zhang C.Y., Kwok S., 1994, ApJ 422, 205.
\item   Arnaud K., Borkowski K.J., Harrington J.P., 1996, ApJ Lett. 462, L75.
\item   Chu Y.-H., Ho Ch.-H., 1995, ApJ Lett. 448, L127.
\item   Leahy D.A., Zhang C.Y., Volk K., Kwok S., 1996, ApJ 466, 352.
\item   Chu Y.-H., Gruendl R.A., Conway G.M., 1998, ApJ 116, 1882.
\item   Chu Y.-H., Kwitter K.B., Kaler J.B., 1993, AJ 106, 650.
\item   Hoare M.G., Barstow M.A., Werner K., Fleming T.A., 1995, MNRAS 273, 812.
\item   Chu Y.-H., Chang T.H., Conway G.M., 1997, ApJ 482, 891.
\item   Leahy D., Kwok S., Yin D., 1998, Bull. AAS 30, 895. (192th Meeting)
\item   Apparao K.M.V., Tarafdar S.P., 1989, ApJ 344, 826.
\item   Apparao K.M.V., Berthiaume G.D., Nousek J.A., 1992, ApJ 397, 534.
\item   Werner K., Motch C., Pakull M., 1993, IAU Symp.155 (eds. R. Weinberger,
      A. Acker), 496.
\end{enumerate}
}
\normalsize
\subsection*{Misclassified PNe}

It is well known that PNe are a rather heterogeneous group of em.-line objects having different morphology or being star-like, especially when we consider their evolutionary stage. There are altogether 245 objects which are no longer classified as PNe (Table 3 and Table 1).\\

It is relatively easy to recognize extragalactic objects (ad 1) according to their morphology on direct photographs or even better from the red-shift of the spectral lines.\\

As to the galactic objects the largest group of misclassified PNe are actually red stars (ad 2) with strong continua near H$\alpha$.~The red stellar continua simulate the existence of emission lines H$\alpha$ and [NII] on low-dispersion spectra. Objects with no emission lines (ad 5) belong also to this category. The reflection nebulae (ad 5) may in principle also be objects in the evolutionary phase before PNe, i.e. pre-PNe. Galactic objects with emission lines are most similar to PNe. Several types of em.-line stars (ad 3) without nebulae look like young PNe; symbiotic stars (ad 6) as well as compact HII regions (ad 4) can be extremely similar to PNe. Also novae with their nebulae could match PNe.\\

The last category represents objects which are actually not real nebulae (ad 7,8) or objects which are identical with already known PNe or their condensations (ad 9).\\

Four objects given also in Table 3 (MaC 1-1, 217$-$00.1*; Cn 1-1, 330+04.1*; PC 11, 331$-$05.1*; M 1-27, 356$-$02.2*) were once removed from the list of PNe as em.-line stars but have now been included again in the list of PNe.\\

The theoretical definition of ``what is a planetary nebula'' plays a great role for the decision as to whether we include the object in the list of PNe or not. This definition depends of course on the present state of knowledge.\\	

{\bf Former confusion with another object-type:               \hfill    N~~~~~~~}
 
\begin{enumerate}
\item  galaxy, em.-line galaxy                          \hfill          53~~~~~~~~~
\item  red star                                         \hfill          53~~~~~~~~~
\item  em.-line star (H$\alpha$ em. only, Be, BQ[~], WR, OB,
                peculiar)                                 \hfill        40~~~~~~~~~
\item  HII region                                         \hfill        27~~~~~~~~~
\item  object with no em. lines, reflection nebula        \hfill        24~~~~~~~~~
\item  symbiotic star                                     \hfill        20~~~~~~~~~
\item  plate fault, nova                                  \hfill        17~~~~~~~~~
\item  several faint stars, no FC and no observations     \hfill         4~~~~~~~~~
\item  identity with another PN or a knot in another object \hfill       7~~~~~~~~~
\end{enumerate}
%\newpage
\\

\vspace{0.8cm}
\normalsize

\subsection*{\textbf{ACKNOWLEDGEMENTS:}}\\

\vspace{.4cm}
   I am very much obliged to Mr.D.K\"uhl for the LATEX-file of
the manuscript, for his general help with the tables and the
charts, as well as for measuring with me the coordinates of 
several hundreds of objects. Without his contribution the 
Catalogue could not have been published in the present form.

   My special thanks are also dedicated to Dr.H.Hagen for the
possibility of using his image processing software, especially for coordinate grids on the
charts and the options for measuring coordinates. 
I am much indebted to Dr.D.Engels for valuable discussions and 
 for his support in various fields. I would 
also like to thank Mrs.A.M\"uller for her help concerning the 
lists of coordinates used and Ms.S.-B.K\"uhl for her valuable work with the finding charts. I acknowledge with thanks the help
of Dr.N.Christlieb with the IRAS sources including the 
corresponding computing programme and Mrs.P.Halilhod\v zi\'c
who also contributed much to the programming. I am indebted to Prof.Dr.J.Schmitt for the discussion on X-ray sources and to Dr.J.-U.Ness for his help also concerning this category of objects. I am very much obliged to Dr.L.Perek
for reading the manuscript and for his constant interest 
concerning the Catalogue. I am thankful to Dr.W.Martin for the critical reading of the text and for his comments. I also wish to thank 
Prof.Dr.J.Hazlehurst and Mrs.Ch.Kohoutek for correcting the English
version of the manuscript. I appreciate the
support of this work by several other colleagues and last but not least the support by the
Directorate of the Hamburg Observatory.
\\

\newpage
I used 714 finding charts, mainly of the Second Digital Sky Survey (DSS2) and
I thank all institutions which are engaged in this project:\\
530 charts are based on southern plates which belong to 
the Second Epoch Survey of the southern sky made by the    
Anglo-Australian Observatory (AAO) with the UK Schmidt Telescope.
These plates have been digitized and compressed by the ST
ScI. The digitized images are copyright (c) 1993-8 by the        
Anglo-Australian Observatory Board, and are distributed herein by
agreement. All Rights Reserved.\\                                  
161 charts are extracted from
the compressed files of the Palomar Observatory - Space Telescope 
Science Institute Digital Sky Survey of the northern sky, based on
scans of the Second Palomar Sky Survey and are copyright (c) 1993-1995 
by the California Institute of Technology and are distributed herein
by agreement.  All Rights Reserved.                              
Produced under Contract No. NAS5-2555 with the National Aeronautics'
and Space Administration.\\
Some charts are based on\\                                         
        (a) photographic data obtained using The UK Schmidt Telescope.    
        The UK Schmidt Telescope was operated by the Royal Observatory         
        Edinburgh, with funding from the UK Science and Engineering Research   
        Council, until 1988 June. Original plate material is copyright (c) 
        the Royal       
        Observatory Edinburgh and the Anglo-Australian Observatory.  The       
        plates were processed into the present compressed digital form with    
        their permission.  The Digitized Sky Survey was produced at the Space  
        Telescope Science Institute under US Government grant NAG W-2166;\\      
        (b) photographic data of the National Geographic Society -- Palomar
        Observatory Sky Survey (NGS-POSS) obtained using the Oschin Telescope of
        Palomar Mountain.  The NGS-POSS was funded by a grant from the National
        Geographic Society to the California Institute of Technology.  The     
        plates were processed into the present compressed digital form with    
        their permission.\\                                         
I obtained the data as fits-files from the Online Digitized Sky Surveys (DSS1 \& 2) server at the ESO/ST-ECF Archive, using the webpage of ESO.\\
 
I thankfully acknowledge the database SIMBAD operated at CDS (Strasbourgh, France).\\


\newpage
\subsection*{ERRATA to CGPN(1967) and to the SUPPLEMENTS:}

Several corrections have already been published in the years 1967 and 1969 (L. Perek, L. Kohoutek, 1967, Bull. Astron. Inst. Czech. 18, 252; 1969, ibid. 20, 381). We include them in the following list which contains the errata to CGPN(1967) as well as to the supplements S1-S5. ``New FC'' means a new finding chart
in CGPN(2000) because the chart given in CGPN(1967) indicates the respective object in the wrong position (in most cases wrong in the original chart from the discoverer or in the literature), or there is no identification in CGPN(1967) at all.\\

\noindent\textbf{CGPN(1967):}\\

\begin{tabular}{rllll}
Page     &Design./Neb.     & Column       & Instead of   & Read	                        \\  
\hline  	        	       		      				    
         &                 &              &              &                              \\  
 15~~    &                 & Surface      & H mag/cir.1' & H mag/cir.1'	                \\  
         &                 & brightnesses & =7.57$-$logS & =7.57$-$\textbf{2.5}logS 	\\  
 28~~    &                 & Ha           & Haro G.,1952,& Haro G.,1952,            	\\  
         &                 &              & Bol.Ton.y T.1& Bol.Ton.y T.1	        \\  
         &                 &              & No.1, 93.    & No.1, \textbf{9}.		\\  
 44~~    &  He 2-248~~~~~~     & Design.~~~~~~~~~~~      & 341$-$29.1~~~~~~~~~~~~& 341$-$\textbf{9}.1	\\  
 44~~    &  Me 2-1         & Design.      & 342$+$17.1   & 342$+$\textbf{2}7.1          \\  
 44~~    &  NGC 6072       & Design.      & 342$+$0.1    & 342$+$\textbf{1}0.1		\\  
 48~~    &  356$+$2.1      & Name         & Th 3$-$31    & Th 3$-$\textbf{13}		\\  
 68~~    &  55$-$1.1       & Name         & K 3$-$42     & K 3$-$4\textbf{3}		\\  
 74~~    &  144$-$15.1     & bII          & $+$15.54     & {\bf{$-$}}15.54		\\  
 74~~    &  171$-$25.1     & Name         & Ba I         & Ba \textbf{1}		\\  
 78~~    &  190$-$17.1     & Rad.velocity & $-$234       & \textbf{$-$23.4}               \\  
 86~~    &  229$-$2.1      & Position     & 8 08.0       & \textbf{7} 08.0		\\  
 91~~    &  261$+$8.1      & Spectrum     & 4340-6717 2  & 4340-6717 \textbf{8}		\\  
104~~    &  318$+$41.1     & Dimensions   & W 61a        & \textbf{VV} 61a		\\  
107~~    &  Cn 1-2         & Magnitude    & $-$11.10 F H$\beta$  & $-$11.\textbf{0}0 F H$\beta$		\\  
118~~    &  352$+$5.1      & Rad.velocity &   0 Mi 57    & \textbf{$+$32} Mi 57         \\  
         &                 &              & $+$32 Ma 64  & \textbf{0} Ma 64		\\  
122~~    &  355$-$3.2      & Name         & H 2 23       & H 2\textbf{-}23		\\  
126~~    &  358$+$7.1      & Position     & 17 08 33.9   & 17 0\textbf{9} 33.9		\\  
128~~    &  Th 3-26        & Design.      & 368$+$3.8    & 3\textbf{5}8$+$3.8		\\  
132~~    &  359$+$3.2      & Position     & 18 22 56     & 1\textbf{7} 22 56		\\  
138~~    &  1$-$1.1        & Position     & 1960 Mi 59   & 19\textbf{5}0 Mi 59          \\  
140~~    &  2$-$3.1        & Position     & $-$28 12 49  & $-$28 \textbf{21} 49		\\  
142~~    &  2$-$7.1        & Position     & 1960 Koal    & 19\textbf{5}0 Koal		\\  
146~~    &  4$+$6.1        & Position     & $-$21 26     & $-$21 26 28		        \\  
148~~    &  5$+$5.1        & Position     & $-$21 30 24  & $-$21 30 \textbf{3}4		\\  
152~~    &  7$-$6.1        & Position     & 1960 Koal    & 19\textbf{5}0 Koal		\\  
159~~    &  11$-$9.1       & Spectrum     & VVal 45      & \textbf{Vy}al 45             \\  
160~~    &  13$-$3.1       & Position     & 18 22 33.3   & 18 2\textbf{6} 33.3		\\  
166~~    &  25$-$17.1      & Position     & 19 36 42.4   & 19 36 \textbf{04.24}		\\  

\end{tabular}
\newpage
\begin{tabular}{rllll}
172~~    &  33$-$2.1       & Position     & 18 54 28.56  & 18 5\textbf{7} 28.56		\\  
184~~    &  49$+$2.1       & Dimensions   & Ko 46d       & Ko \textbf{64}d		\\  
198~~    &  68$+$1.2       & Position     & 19 47 24     & 19 \textbf{5}7 24		\\  
205~~    &  86$-$8.1       & Detailed inv.& exp.8        & ex\textbf{c}.8               \\  
220~~    &  159$-$15.1     & Rad.velocity & $+$10.3      & \textbf{$-$}10.3		\\  
242~~    &  107$+$2.1      & Name         & 7375         & 73\textbf{54}		\\  
253~~    &  6$+$2.5        & Name         & M 1-13       & M 1-\textbf{31}		\\  
         &    (17:49.7)    &	          &		 &				\\  
256~~    &  107$+$21.1     & Name         & K 1-26       & K 1-\textbf{6}               \\  
         &    (20:05.3)    &	          &		 &				\\  
256~~    &  K 4-37~~~~~~~~ & Design.~~~~~~~~~~~& 62$+$2.1~~~~~~~~~~~~~~& 6\textbf{6}$+$2.1 \\  
         &    (19:49.0)    & 	          &		 &				\\  
265~~    &  He 2-131       & Design.      & 315$-$3.1    & 315$-$\textbf{1}3.1		\\  
266~~    &  He 2-262       & Name         &\textbf{H 2-16}& --			        \\  
266~~    &  He 2-261       & Name         & --           &\textbf{H 2-16}               \\  
274~~    &  Th 1-4         &              & Pe 1-21      &\textbf{Pe 1-21}		\\  
Pl.6~    &  249$-$5.1      & A 23         & no ident.    &\textbf{new FC}	        \\  
   10    &  265$-$2.1      & Ve 26        &              &\textbf{new FC}		\\  
   12    &  274$+$2.1      & He 2-34      & id. uncert.  &\textbf{new FC}		\\  
   15    &  285$+$1.1      & He 2-52      & id. uncert.  &\textbf{new FC}		\\  
   16    &  288$-$0.1      & Hf 39        &              &\textbf{new FC}               \\  
   30    &  324$-$1.1      & He 2-133     &              &\textbf{new FC}		\\  
   32    &  326$-$10.1     & Cn 1-2       &              &\textbf{new FC}		\\  
   46    &  351$-$5.1      & He 2-275     &              & *)     			\\  
   47    &  353$+$8.1      & MyCn 26      &              &\textbf{new FC}		\\  
   49    &  356$-$0.1      & Th 3-34      &              &\textbf{new FC}               \\   
   53    &  357$-$4.3      & H 1-43       &              &\textbf{new FC}               \\
   55    &  358$+$1.4      & Bl B         & id. uncert.  &\textbf{new FC}		\\  
   55    &  358$-$2.1      & M 4-7        &              &\textbf{new FC}		\\  
   55    &  358$-$2.2      & Bl 3-6       &              & *)     			\\  
   61    &  1$+$5.1        & H 1-14       &              &\textbf{new FC}		\\  
   63    &  2$+$1.1        & H 2-20       &              &\textbf{new FC}		\\  
   63    &  2$-$1.1        & Pe 2-11      &              &\textbf{new FC}               \\  
   69    &   4$-$5.2       & He 2-376     &              &\textbf{new FC}		\\  
   88    &  37$-$5.1       & A 58         &              &\textbf{new FC}		\\  
   98    &  59$-$1.1       & He 1-3       &              &\textbf{new FC}		\\  
   105   &  72$-$17.1      & A 74         & no ident.    &\textbf{new FC}  		\\  
   112   &  144$+$6.1      & 1501         & scale 1'     & scale \textbf{5}'		\\  
Pl.113   &  164$+$31.1     & 2474-5       & 2474-5       &\textbf{JnEr 1}               \\  
         &                 &              &              &                              \\  
\hline
         &                 &              &              &                              \\  
\multicolumn{5}{l}{*) Correct finding chart in Acker et al. (1988).}                    \\
\end{tabular}

\newpage
\noindent\textbf{SUPPLEMENTS:}\\

\noindent\textbf{Supplement 1} (Proc. IAU Symp. No.76, p.47; 1978):  \\

All names of PNe were printed in capital letters. This was therefore a mistake in some cases. For this reason instead of 

\hspace{2cm}AE, CN, HE, KR, LO, SA, SH, SUWT, WE\\
read

\hspace{2cm}\textbf{Ae, Cn, He, Kr, Lo, Sa, Sh, SuWt, We}\\
in the text, in Tables 1 and 2, in the Appendix to Table 1 as well as in the corresponding remarks.\\ 

\noindent\textbf{Supplement 2} (Proc. IAU Symp. No.103, p.17; 1983):   \\
\begin{tabular}{lllll}
      &           &             &            &                          \\
Page  &  Design.  &  Name       &Instead of  &  Read			\\
\hline
 18   &           &  Pu 2       &175+6.1~~~~~~~~~~~~~~     & \textbf{173+3.1}		\\
      &           &  Pu 2       &5 59.20     &  5 \textbf{3}9.20	\\
 18   &  217$+$2.1&             &Sp 3-1      &  S\textbf{P} 3-1		\\
 20   &  31$-$0.1 &             &$-$0 06.8   &  $-$\textbf{1} 06.8	\\
 22   &           &  References to Table 1~~ &Sp 3        &  S\textbf{P} 3		\\
 23   &  65$-$27.2&             &CiPg        & \textbf{CIPG}		\\
 23   &           &  NGC 1985   &176$-$0.1   &  176\textbf{$+$}0.1	\\
      &           &             &            &                          \\
\multicolumn{5}{l}{\textbf{Supplement 3} (Proc. IAU Symp. No.131, p.29; 1989):}  \\
      &           &             &            &                          \\
 31   &  11$+$17.1*& ~~~~~~~~~~~~~~~~~       &  DeHt 1     &   DeHt \textbf{10}  \\
      &           &             &            &                          \\
\multicolumn{5}{l}{\textbf{Supplement 4} (Proc. IAU Symp. No.155, p.36; 1992; Astron. Nachr. 315, 63; 1994):}\\
      &           &             &            &                          \\
 65   &  9$-$07.1 &    Remarks  &  NGC 6566  &   NGC 6\textbf{65}6      \\
\end{tabular}

\newpage
\subsection*{\textbf{REFERENCES:}}

Acker A., 1978, A\&AS 33, 367.\\   
Acker A., Lundstr\"om I., Stenholm B., 1988, A\&AS 73, 325.\\
Acker A., Ochsenbein F., Stenholm B., Tylenda R., Marcout J., Schohn C., 1992, Strasbourg-
\hspace*{0.8cm}   ESO Catalogue of Galactic Planetary Nebulae, European Southern Observatory.\\  
\hspace*{0.8cm}   (SECGPN)\\
Acker A., Marcout J., Ochsenbein F., 1996, First Supplement to the SECGPN, Observatoire 
\hspace*{0.8cm}   de Strasbourg.  (S1 to SECGPN)\\
Allen D.A., 1984, Proc. Astron. Soc. Austr. 5, 369.\\
Aller L.H., 1956, Gaseous Nebulae, p.65, London.\\
Becker R.H., White R.L., Helfand D.J., Zoonemathermani S., 1994, ApJS 91, 347.\\
Belczy\'nski K., Miko\l ajewska J., Munari U. et al., 2000, A\&AS 146, 407.\\
Bl\"ocker T., 1995, A\&A 299, 755.\\
Bond H.E., 2000, ASP Conference Series, Vol.199, 115.\\
Ciardullo R., Bond H.E., Sipior M.S. et al., 1999, AJ 118, 488.\\ 
Corradi R.L.M., 1995, MNRAS 276, 521.\\
Curtis H.D., 1918, Publ. Lick Obs. 13, 55.\\
Harman R.J., Seaton M.J., 1964, ApJ 140, 824.\\
Kholopov P.N. (ed) et al., 1982, New Catalogue of Suspected Variable Stars, Moscow,\\
\hspace*{0.8cm}    ``Nauka''.  (NSV)\\
Kholopov P.N. (ed.) at al., Samus' N.N. (ed.) et al., 1985-1990, General Catalogue of 
\hspace*{0.8cm}     \mbox{Variable Stars} (4th edition), I.-VI., Moscow, ``Nauka''.  (GCVS)\\
Kistiakowsky V., Helfand D.J., 1995, AJ 110, 2225.\\
Kohoutek L., 1978, in Proc. IAU Symp. No.76, p.47 (ed. Y. Terzian), D. Reidel Publ. Co., 
\hspace*{0.8cm}    Dordrecht, Boston. (\textbf{Supplement 1 to CGPN})\\
Kohoutek L., 1983, in Proc. IAU Symp. No.103, p.17 (ed. D.R. Flower), D. Reidel Publ. 
\hspace*{0.8cm}     Co., Dordrecht, Boston, London. (\textbf{Supplement 2 to CGPN})\\
Kohoutek L., 1989, in Proc. IAU Symp. No.131, p.29 (ed. S.Torres-Peimbert), Kluwer 
\hspace*{0.8cm}     Academic Publ., Dordrecht, Boston, London. (\textbf{Supplement 3 to CGPN})\\
Kohoutek L., 1992, in Proc. IAU Symp. No.155, p.36 (eds. A.Acker, R. Weinberger), 
\hspace*{0.8cm}     Kluwer Academic Publ., Dordrecht, Boston, London - abstract; Astron. Nachr. 315, 
\hspace*{0.8cm}     (1994), 63. (\textbf{Supplement 4 to CGPN})\\
Kohoutek L., 1994, Astron. Nachr. 315, 235.\\
Kohoutek L., 1997, in Proc. IAU Symp. No.180, p.23 (eds.H.J.Habing, H.J.G.L.M.Lamers), 
\hspace*{0.8cm}     Kluwer Academic Publ., Dordrecht, Boston, London - abstract; Astron. Nachr. 318\\
\hspace*{0.8cm}     (1997), 35. (\textbf{Supplement 5 to CGPN})\\
Kohoutek L., 2000, \textbf{Supplement 6 to CGPN} - available only from the author.\\
Kohoutek L., K\"uhl D., 1999, private comm., in preparation.\\
Kwok S., 1982, ApJ 258, 280.\\
Kwok S., Purton C.R., FitzGerald P.M., 1978, ApJ Lett. 219, L125.\\
Manchado A., Guerrero M.A., Stanghellini L., Serra-Ricart M., 1996, The IAC Morphologi-
\hspace*{0.8cm}     cal Catalog of Northern Galactic Planetary Nebulae, Instituto de Astrof�sica de \\
\hspace*{0.8cm}     Canarias.\\
Paczy\'nski B., 1971, Acta Astron. 21, 417.\\
Page T., 1942, ApJ 96, 78.\\
Parker Q.A., Phillips S., Morgan D.H., 1999, ASP Conference Ser. 168, 126.\\
Perek L., Kohoutek L., 1967, Catalogue of Galactic Planetary Nebulae, Academia Prague.   
\hspace*{0.8cm}     (\textbf{CGPN1967})\\
Pottasch S.R., 1996, A\&A 307, 561.\\
Pottasch S.R., Bignell C., Olling R., Zijlstra A.A., 1988, A\&A 205, 248.\\
Preite-Martinez A., 1988, A\&AS 76, 317.\\
Sabbadin F., 1986, A\&AS 64, 579.\\ 
Sanduleak N., Stephenson C.B., 1973, ApJ 185, 899 (see also 1972, ApJ 178, 183).\\
Schmid H.M., 1989, A\&A 211, L31\\
Sch\"onberner D., 1979, A\&A 79, 108.\\
Schwarz H.E., Corradi R.L.M., 1992, A\&A 265, L37.\\
Schwarz H.E., Aspin C., Lutz J.H., 1989, ApJ Lett. 344, L29.\\
Shklovsky I.S., 1956, Astron. Zh. 33, 315.\\
Tweedy R.W., Kwitter K.B., 1996, ApJS 107, 255.\\
Vorontsov-Velyaminov B.A., 1947, Astron. Zh. 24, 83 (see also Astron. Tsirk. Moscow\\ 
\hspace*{0.8cm}No.55, 7, 1946).\\
Vorontsov-Velyaminov B.A., 1948, Gaseous Nebulae and Novae (in Russian), Moscow \\
\hspace*{0.8cm}(Gasnebel und Neue Sterne, Berlin 1953, p.600, Fig.97).\\
Vorontsov-Velyaminov B.A., 1962, Mitt. Astron. Sternberg-Inst. Moscow, No.118, 3.\\
Vorontsov-Velyaminov B.A., Parenago P.P., 1931, Astron. Zh. 8, 206.\\

\vspace{1.cm}
{\bf The correct spelling of some names used in the tables:} \\

\begin{tabular}{ll}                           
  B\'atiz G.                    &  K\"ustner F.           \\
  Belczy\'nski K.               &  Lundstr\"om I.         \\
  B\"ohm-Vitense E.             &  Mar\v s\'alkov\'a P.   \\
  Cant\'o J.                    &  Nordstr\"om B.         \\
  Clari\'a J.J.                 &  Paczy\'nski B.         \\
  Garc\'\i a-Lario P.           &  P\v ekn\'y Z.          \\
  G\"otz W.                     &  Pe\~na M.      ~~~~~~~~\\
  Gonz\'alez L.E.               &  Rodr\'\i guez L.F.     \\
  Gr\"obner H.                  &  Ru\'\i z M.T.          \\
  Guti\'errez-Moreno A.~~~~~~~~~~~~~~~~~ &  S\'anchez G.           \\
  Kohlsch\"utter A.             &  Sangu\'\i n J.G.       \\
  K\"uhl D.                     &  Sch\"onberner D.       \\
\end{tabular}			                          

\end{document}
            